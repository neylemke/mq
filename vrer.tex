\thispagestyle{headandfoot}
\begin{center} {\large Verifica��o I}
\end{center}
\vspace{0.5cm} Nome:\rule{14cm}{0.01cm} \\



\vspace{1 cm}
 
{\bf S\'o ser\~ao aceitas respostas devidamente justificadas.}

\vspace{1 cm}
\begin{questions}

\question[2.0] Considere uma part�cula em um po�o infinito localizado
entre 0 e $a$, descrito
pela fun��o de onda em $t=0$:
$$\psi(x,0)= A x (x-a/2) (x-a)$$ 

se $x>0$ e $x<a$ e zero nos demais casos. 
\begin{parts}
  \item Determine o valor da constante de normaliza��o $A$
  \item Calcule a fun��o de onda no espa�o de momento. 
  \item Calcule o valor esperado da posi��o  em $t=0$.
  \item Qual � a probabilide em uma medida de energia encontrar o
    sistema no primeiro estado excitado. 
\end{parts}

\question[2.0] Considere a fun��o:

$$\psi(x,t)=A e ^{-a (m x^2/\hbar +i t)}$$

\begin{parts}
  \item Determine $A$.
  \item Para qual fun��o de energia potencial $V(x)$ $\psi$ satisfaz a
    equa��o de Schr�dinger. 
\end{parts}





\question[2.0] Considere os autoestados coerentes do oscilador harm�nico em
uma dimens�o:

$$a^\dagger |\alpha\rangle=\alpha|\alpha\rangle$$

Calcule:

\begin{parts}
\item $\langle \alpha | x | \alpha \rangle$
\item $\langle \alpha | x^2 | \alpha \rangle$
\end{parts}

\question[2.0] Considere o vetor 
$$|\psi\rangle=\alpha \left(-1|10\rangle- 3 |1-1\rangle+2|11\rangle\right)$$
onde os vetores $|lm\rangle$ representam os 
autoestados de momento angular.  Calcule:
\begin{parts}
\item $\alpha$ para que o vetor esteja normalizado.
\item $\langle L^2 \rangle$
\item $\langle L_x \rangle$
\item qual � a probabilidade de ser medido $L_z=-\hbar$.
\end{parts}

\question[2.0] Calcule $\langle r \rangle$ e $\langle r ^2\rangle$  
para um �tomo no estado fundamental do �tomo de hidrog�nio.  


\end{questions}





%%% Local Variables: 
%%% mode: latex
%%% TeX-master: "exame"
%%% TeX-master: "exame"
%%% TeX-master: "exame"
%%% TeX-master: "exame"
%%% End: 
%%% End: 
