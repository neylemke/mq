
% $Header: /cvsroot/latex-beamer/latex-beamer/examples/beamerexample5.tex,v 1.22 2004/10/08 14:02:33 tantau Exp $

\documentclass[11pt]{beamer}

\usetheme{Darmstadt}

\usepackage{times}
\usefonttheme{structurebold}

%\usepackage[english]{babel}
\usepackage[portuges]{babel}
\usepackage{pgf,pgfarrows,pgfnodes,pgfautomata,pgfheaps}
\usepackage{amsmath,amssymb}
\usepackage[latin1]{inputenc}
\usepackage{graphicx}

\setbeamercovered{dynamic}

\newcommand{\Lang}[1]{\operatorname{\text{\textsc{#1}}}}

\newcommand{\Class}[1]{\operatorname{\mathchoice
  {\text{\sf \small #1}}
  {\text{\sf \small #1}}
  {\text{\sf #1}}
  {\text{\sf #1}}}}

\newcommand{\NumSAT}      {\text{\small\#SAT}}
\newcommand{\NumA}        {\#_{\!A}}

\newcommand{\barA}        {\,\bar{\!A}}

\newcommand{\Nat}{\mathbb{N}}
\newcommand{\Set}[1]{\{#1\}}

\pgfdeclaremask{tu}{beamer-tu-logo-mask}
\pgfdeclaremask{computer}{beamer-computer-mask}
\pgfdeclareimage[interpolate=true,mask=computer,height=2cm]{computerimage}{beamer-computer}
\pgfdeclareimage[interpolate=true,mask=computer,height=2cm]{computerworkingimage}{beamer-computerred}
\pgfdeclareimage[mask=tu,height=.5cm]{logo}{logounesp}

\logo{\pgfuseimage{logo}}

\title{Introdu\c{c}\~ao}
\author{Ney Lemke}
\institute[IBB-UNESP]{%
    Mec\^anica Qu\^antica }
%%%%%    Departamento de F\~AƒÂƒ\~A‚­sica e Biof\~AƒÂƒ\~A‚­sica}
\date{2012}                                

\colorlet{redshaded}{red!25!bg}
\colorlet{shaded}{black!25!bg}
\colorlet{shadedshaded}{black!10!bg}
\colorlet{blackshaded}{black!40!bg}

\colorlet{darkred}{red!80!black}
\colorlet{darkblue}{blue!80!black}
\colorlet{darkgreen}{green!80!black}

\def\radius{0.96cm}
\def\innerradius{0.85cm}

\def\softness{0.4}
\definecolor{softred}{rgb}{1,\softness,\softness}
\definecolor{softgreen}{rgb}{\softness,1,\softness}
\definecolor{softblue}{rgb}{\softness,\softness,1}

\definecolor{softrg}{rgb}{1,1,\softness}
\definecolor{softrb}{rgb}{1,\softness,1}
\definecolor{softgb}{rgb}{\softness,1,1}

\newcommand{\Bandshaded}[2]{
  \color{shadedshaded}
  \pgfmoveto{\pgfxy(-0.5,0)}
  \pgflineto{\pgfxy(-0.6,0.1)}
  \pgflineto{\pgfxy(-0.4,0.2)}
  \pgflineto{\pgfxy(-0.6,0.3)}
  \pgflineto{\pgfxy(-0.4,0.4)}
  \pgflineto{\pgfxy(-0.5,0.5)}
  \pgflineto{\pgfxy(4,0.5)}
  \pgflineto{\pgfxy(4.1,0.4)}
  \pgflineto{\pgfxy(3.9,0.3)}
  \pgflineto{\pgfxy(4.1,0.2)}
  \pgflineto{\pgfxy(3.9,0.1)}
  \pgflineto{\pgfxy(4,0)}
  \pgfclosepath
  \pgffill

  \color{black}  
  \pgfputat{\pgfxy(0,0.7)}{\pgfbox[left,base]{#1}}
  \pgfputat{\pgfxy(0,-0.1)}{\pgfbox[left,top]{#2}}
}

\newcommand{\Band}[2]{
  \color{shaded}
  \pgfmoveto{\pgfxy(-0.5,0)}
  \pgflineto{\pgfxy(-0.6,0.1)}
  \pgflineto{\pgfxy(-0.4,0.2)}
  \pgflineto{\pgfxy(-0.6,0.3)}
  \pgflineto{\pgfxy(-0.4,0.4)}
  \pgflineto{\pgfxy(-0.5,0.5)}
  \pgflineto{\pgfxy(4,0.5)}
  \pgflineto{\pgfxy(4.1,0.4)}
  \pgflineto{\pgfxy(3.9,0.3)}
  \pgflineto{\pgfxy(4.1,0.2)}
  \pgflineto{\pgfxy(3.9,0.1)}
  \pgflineto{\pgfxy(4,0)}
  \pgfclosepath
  \pgffill

  \color{black}  
  \pgfputat{\pgfxy(0,0.7)}{\pgfbox[left,base]{#1}}
  \pgfputat{\pgfxy(0,-0.1)}{\pgfbox[left,top]{#2}}
}

\newcommand{\BaenderNormal}
{%
  \pgfsetlinewidth{0.4pt}
  \color{black}
  \pgfputat{\pgfxy(0,5)}{\Band{input tapes}{}}
  \pgfputat{\pgfxy(0.35,4.6)}{\pgfbox[center,base]{$\vdots$}}
  \pgfputat{\pgfxy(0,4)}{\Band{}{}}

  \pgfxyline(0,5)(0,5.5)
  \pgfxyline(1.2,5)(1.2,5.5)
  \pgfputat{\pgfxy(0.25,5.25)}{\pgfbox[left,center]{$w_1$}}

  \pgfxyline(0,4)(0,4.5)
  \pgfxyline(1.8,4)(1.8,4.5)        
  \pgfputat{\pgfxy(0.25,4.25)}{\pgfbox[left,center]{$w_n$}}
  \ignorespaces}

\newcommand{\BaenderZweiNormal}
{%
  \pgfsetlinewidth{0.4pt}
  \color{black}
  \pgfputat{\pgfxy(0,5)}{\Band{Zwei Eingabeb\~AƒÂƒ\~A‚¤nder}{}}
  \pgfputat{\pgfxy(0,4.25)}{\Band{}{}}

  \pgfxyline(0,5)(0,5.5)
  \pgfxyline(1.2,5)(1.2,5.5)
  \pgfputat{\pgfxy(0.25,5.25)}{\pgfbox[left,center]{$u$}}

  \pgfxyline(0,4.25)(0,4.75)
  \pgfxyline(1.8,4.25)(1.8,4.75)        
  \pgfputat{\pgfxy(0.25,4.5)}{\pgfbox[left,center]{$v$}}
  \ignorespaces}

\newcommand{\BaenderHell}
{%
  \pgfsetlinewidth{0.4pt}
  \color{black}
  \pgfputat{\pgfxy(0,5)}{\Bandshaded{input tapes}{}}
  \color{shaded}
  \pgfputat{\pgfxy(0.35,4.6)}{\pgfbox[center,base]{$\vdots$}}
  \pgfputat{\pgfxy(0,4)}{\Bandshaded{}{}}

  \color{blackshaded}
  \pgfxyline(0,5)(0,5.5)
  \pgfxyline(1.2,5)(1.2,5.5)
  \pgfputat{\pgfxy(0.25,5.25)}{\pgfbox[left,center]{$w_1$}}

  \pgfxyline(0,4)(0,4.5)
  \pgfxyline(1.8,4)(1.8,4.5)        
  \pgfputat{\pgfxy(0.25,4.25)}{\pgfbox[left,center]{$w_n$}}
  \ignorespaces}

\newcommand{\BaenderZweiHell}
{%
  \pgfsetlinewidth{0.4pt}
  \color{black}
  \pgfputat{\pgfxy(0,5)}{\Bandshaded{Zwei Eingabeb\~AƒÂƒ\~A‚¤nder}{}}%
  \color{blackshaded}
  \pgfputat{\pgfxy(0,4.25)}{\Bandshaded{}{}}
  \pgfputat{\pgfxy(0.25,4.5)}{\pgfbox[left,center]{$v$}}
  \pgfputat{\pgfxy(0.25,5.25)}{\pgfbox[left,center]{$u$}}%

  \pgfxyline(0,5)(0,5.5)
  \pgfxyline(1.2,5)(1.2,5.5)

  \pgfxyline(0,4.25)(0,4.75)
  \pgfxyline(1.8,4.25)(1.8,4.75)        
  \ignorespaces}

\newcommand{\Slot}[1]{%
  \begin{pgftranslate}{\pgfpoint{#1}{0pt}}%
    \pgfsetlinewidth{0.6pt}%
    \color{structure}%
    \pgfmoveto{\pgfxy(-0.1,5.5)}%
    \pgfbezier{\pgfxy(-0.1,5.55)}{\pgfxy(-0.05,5.6)}{\pgfxy(0,5.6)}%
    \pgfbezier{\pgfxy(0.05,5.6)}{\pgfxy(0.1,5.55)}{\pgfxy(0.1,5.5)}%
    \pgflineto{\pgfxy(0.1,4.0)}%
    \pgfbezier{\pgfxy(0.1,3.95)}{\pgfxy(0.05,3.9)}{\pgfxy(0,3.9)}%
    \pgfbezier{\pgfxy(-0.05,3.9)}{\pgfxy(-0.1,3.95)}{\pgfxy(-0.1,4.0)}%
    \pgfclosepath%
    \pgfstroke%
  \end{pgftranslate}\ignorespaces}

\newcommand{\SlotZwei}[1]{%
  \begin{pgftranslate}{\pgfpoint{#1}{0pt}}%
    \pgfsetlinewidth{0.6pt}%
    \color{structure}%
    \pgfmoveto{\pgfxy(-0.1,5.5)}%
    \pgfbezier{\pgfxy(-0.1,5.55)}{\pgfxy(-0.05,5.6)}{\pgfxy(0,5.6)}%
    \pgfbezier{\pgfxy(0.05,5.6)}{\pgfxy(0.1,5.55)}{\pgfxy(0.1,5.5)}%
    \pgflineto{\pgfxy(0.1,4.25)}%
    \pgfbezier{\pgfxy(0.1,4.25)}{\pgfxy(0.05,4.15)}{\pgfxy(0,4.15)}%
    \pgfbezier{\pgfxy(-0.05,4.15)}{\pgfxy(-0.1,4.2)}{\pgfxy(-0.1,4.25)}%
    \pgfclosepath%
    \pgfstroke%
  \end{pgftranslate}\ignorespaces}

\newcommand{\ClipSlot}[1]{%
  \pgfrect[clip]{\pgfrelative{\pgfxy(-0.1,0)}{\pgfpoint{#1}{4cm}}}{\pgfxy(0.2,1.5)}\ignorespaces}

\newcommand{\ClipSlotZwei}[1]{%
  \pgfrect[clip]{\pgfrelative{\pgfxy(-0.1,0)}{\pgfpoint{#1}{4.25cm}}}{\pgfxy(0.2,1.25)}\ignorespaces}


\AtBeginSection[]{\frame{\frametitle{Outline}\tableofcontents[current]}}

\begin{document}

\frame{\titlepage}

%\section*{Outline}

\frame{\frametitle{Outline}\tableofcontents} 

 \frame{ 

"All
   things appear and disappear because of the concurrence of causes and
   conditions. Nothing ever exists entirely alone; everything is in
   relation to everything else.''

   The Buddha (Gautama Siddharta, the founder of Buddhism, 563-483
   B.C.)

}
\section{Bases Filos\'oficas}


\frame{\frametitle{Mundo e Observador}
  \begin{tabular}{c c}
    \begin{minipage}{0.45\textwidth}
      {\bf Realismo} Existe uma realidade independente do observador,
      que n\~ao \'e percebida de forma direta pelos sentidos do
      observador.


      {\bf Idealismo} N\~ao existe uma realidade independendente do
      observador, a realidade \'e  um produto da consci\^encia do
      observador.
    \end{minipage}&
    \begin{minipage}{0.45\textwidth}
      \includegraphics[scale=0.5]{magrite1}
    \end{minipage}
  \end{tabular}
}


 \frame{\frametitle{Determinismo}
   \begin{tabular}{c c}
     \begin{minipage}{0.45\textwidth}
       {\bf Determinismo} A natureza obedece a leis imut\'aveis e
       universais, o futuro  \'e consequ\^encia direta do
       passado. Eventos aleat\'orios s\~ao causados pela nossa ignor\^ancia.


       {\bf Probabilismo} O presente n\~ao \'e  consequ\^encia do passado, as
       leis da natureza apenas predizem a probabilidade de ocorr\^encia
       de um determinado evento.
     \end{minipage}&
     \begin{minipage}{0.45\textwidth}
       \begin{center}
         \includegraphics[scale=1]{cause}
       \end{center}
     \end{minipage}
   \end{tabular}


 } 

\frame{\frametitle{Determinismo}
  \begin{tabular}{c c}
    \begin{minipage}{0.45\textwidth}

      {\bf Intervencionismo} Existe uma ou v\'arias entidades que
      interferem diretamente na natureza. Existem mecanismos que
      permitem aos homens interagirem com estas entidades.
    \end{minipage}&
    \begin{minipage}{0.45\textwidth}
      \includegraphics[scale=1.0]{zeus}
    \end{minipage}
  \end{tabular}


}

\frame{\frametitle{Localidade e n\~ao localidade}
  \begin{tabular}{c c}
    \begin{minipage}{0.45\textwidth}
      {\bf Realidade Local} Existem entidades independentes, que
      interagem com outras entidades independentes de forma n\~ao
      instant\^anea.

      {\bf N\~ao localidade} N\~ao existem entidades independentes, a
      natureza \'e formada por uma \'unica entidade, eventos muito
      distantes podem interferir instantaneamente.

    \end{minipage} %&
    \begin{minipage}{0.45\textwidth}
      \begin{center}
        \includegraphics[scale=0.3]{naolocal}
     \end{center}
    \end{minipage}
  \end{tabular}
}


\frame{\frametitle{Teorema de Bell} 

  Ele garante condi\c{c}\~oes bastante
  gerais que teoria f\'isicas: realistas e locais, devem obedecer.  Os
  sistemas f\'isicos em diversas situa\c{c}\~oes violam estas condi\c{c}\~oes, ou
   seja a natureza n\~ao pode ser descrita por uma teoria simultaneamente:

  \begin{itemize}
  \item local e
  \item realista.
  \end{itemize}

  A MQ n\~ao obedece estas condi\c{c}\~oes e descreve os resultados
  experimentais.  Mas podem existir outras teorias mais gerais que ela
  e que possam explicar os mesmos resultados experimentais.
}

\frame{\frametitle{Resultados Recentes}
  \begin{itemize}
  \item Existem evid\^encias experimentais que os estados qu\^anticos
    possuam exist\^encia independente do observador.
  \item O que nos deixaria apenas com a possibilidade que a natureza
    seja n\~ao local.
  \end{itemize}
}

\frame{\frametitle{Vis\~ao de Einstein-Galileo}
  \begin{tabular}{c c}
    \begin{minipage}{0.45\textwidth}
      A natureza \'e descrita por leis relativamente simples que podem
      ser compreendidas de forma intuitiva, ou pelo menos que possam
      ser apreendidas pelo intelecto humano. A MQ \'e  uma teoria
      provis\'oria, que poder\'a ser abandonada quando uma teoria completa
      for descoberta.
    \end{minipage}&
    \begin{minipage}{0.45\textwidth}
      \includegraphics[scale=1.0]{einstein}
    \end{minipage}
  \end{tabular}
}


\frame{\frametitle{Pragmatismo}
  \begin{tabular}{c c}
    \begin{minipage}{0.45\textwidth}

      \begin{itemize}
      \item As teorias s\~ao provis\'orias,
      \item descrevem de forma imperfeita a natureza,
      \item teorias n\~ao s\~ao intuitivas, mas podemos manipul\'a-las de
        maneira formal e realizar predi\c{c}\ ~oes
      \end{itemize}
 
    \end{minipage}&
    \begin{minipage}{0.45\textwidth}
      \includegraphics[scale=0.3]{magritecondhumana1933}
    \end{minipage}
  \end{tabular}

}

\frame{\frametitle{Pragmatismo Dawkins} O nosso c\'erebro foi moldado
  pela evolu\c{c}\~ao e est\'a adaptado para viver no mundo em nossa escala de
  tempo e espa\c{c}o, em escalas diferentes dessas esse aparato funciona
  de forma prec\'aria e deve ser substitu\'ido pelo formalismo matem\'atico.
}

\frame{\frametitle{Mec\^anica Qu\^antica - Conpenhagen}
  \begin{itemize}
  \item Formalimo matem\'atico que permite realizar predi\c{c}\~oes de
    resultados experimentais.
  \item Principal ingrediente $\Psi$: fun\c{c}\~ao de onda n\~ao pode ser
    medida ou percebida de forma direta.
  \item N\~ao determin\'istica e n\~ao local.
  \item Processo de medida interfere com objeto experimental.
  \end{itemize}
}

\frame{\frametitle{Observa\c{c}\~oes}
  \begin{tabular}{c c}
    \begin{minipage}{0.45\textwidth}
      \begin{itemize}
      \item O fato de que o processo de medida interfere com o objeto
        n\~ao implica que o observador ``controle'' este objeto.
      \item objetos com energias altas o suficiente podem ser
        descritos com boa aproxima\c{c}\~ao pela mec\^anica cl\'assica
      \item existem v\'arios efeitos qu\^anticos relacionados a estrutura
        da mat\'eria.
      \end{itemize}
    \end{minipage}&
    \begin{minipage}{0.45\textwidth}
      \includegraphics[scale=0.4, angle=90]{ParticleTracks}
    \end{minipage}
  \end{tabular}
}

\frame{\frametitle{MQ \& Sociedade}
  \begin{itemize}
  \item Transistores
  \item Criptografia Qu\^antica
  \item Luz Laser
  \item $\ldots$
  \end{itemize}
}

\frame{\frametitle{MQ \& Sociedade}
  \begin{itemize}
  \item Homeopatia?
  \item Administra\c{c}\~ao de Empresas?
  \item Cura Qu\^antica?
  \item Telepatia?
  \item Universos Paralelos?
  \item Ovnis?
  \end{itemize}
}


{\frame{\frametitle{Cura Qu\^antica}


 
     Os ``quanta'' s\~ao as mais \'{\i}nfimas part\'{\i}culas (sub
    at\^omicas conhecidas), de energia existentes no Universo, elas s\~ao
    part\'{\i}culas e s\~ao energia, e comportam-se de forma inteligente,
    transpondo os limites da barreira : espa\c{c}o / tempo. 
     Os princ\'{\i}pios da
    F\'{\i}sica Qu\^antica levam-nos a um conceito absolutamente novo, a um
   Paradigma Multidimensional sem limites, onde n\~ao h\'a barreiras nem
    certezas, num infinito Oceano C\'osmico, que \'e um Multidimensional
    campo de potencialidades.

   Nesse vast\'{\i}ssimo campo ou oceano energ\'etico, tudo se relaciona com
   tudo, campo esse, do qual n\'os tamb\'em fazemos parte.  

 Investimento: 150 Euros

{\tt http://reikimaiscuraquantica.blog.com/1660845/}

}

\end{document}


