\thispagestyle{headandfoot}
\begin{center} {\large Verifica��o II}
\end{center}
\vspace{0.5cm} Nome:\rule{14cm}{0.01cm} \\



\vspace{1 cm}
 
{\bf S\'o ser\~ao aceitas respostas devidamente justificadas.}

\vspace{1 cm}
\begin{questions}


\question[2.0] Considere os autoestados coerentes do oscilador harm�nico em
uma dimens�o:

$$a^\dagger |\alpha\rangle=\alpha|\alpha\rangle$$

Calcule:

\begin{parts}
\item $\langle \alpha | x | \alpha \rangle$
\item $\langle \alpha | x^2 | \alpha \rangle$
\end{parts}

\question[3.0] Considere o vetor 
$$|\psi\rangle=\alpha \left(-1|10\rangle- 3 |1-1\rangle+2|11\rangle\right)$$
onde os vetores $|lm\rangle$ representam os 
autoestados de momento angular.  Calcule:
\begin{parts}
\item $\alpha$ para que o vetor esteja normalizado.
\item $\langle L^2 \rangle$
\item $\langle L_x \rangle$
\item qual � a probabilidade de ser medido $L_z=-\hbar$.
\end{parts}

\question[2.0] Calcule $\langle r \rangle$ e $\langle r ^2\rangle$  
para um �tomo no estado fundamental do �tomo de hidrog�nio.  

\question[2.0] Considere um sistema de spin 1/2, voc� realiza uma medida
de $2S_x+2 S_y+S_z$ e obt�m o valor $3 \hbar/2$. Qual � a probabilidade de  
se obter em uma  medida subsequente de $S_y$ $-\hbar/2$? 


\question[2.0] Escreva a matriz que representa o operador $xp$ na base
dos autoestados de energia do oscilador harm�nico. 


\end{questions}





%%% Local Variables: 
%%% mode: latex
%%% TeX-master: "exame"
%%% TeX-master: "exame"
%%% TeX-master: "exame"
%%% TeX-master: "exame"
%%% End: 
