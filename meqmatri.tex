% $Header: /cvsroot/latex-beamer/latex-beamer/examples/beamerexample5.tex,v 1.22 2004/10/08 14:02:33 tantau Exp $

\documentclass[11pt]{beamer}

\usetheme{Darmstadt}

\usepackage{times}
\usefonttheme{structurebold}

%\usepackage[english]{babel}
\usepackage[portuges]{babel}
\usepackage{pgf,pgfarrows,pgfnodes,pgfautomata,pgfheaps}
\usepackage{amsmath,amssymb}
%\usepackage[latin8]{inputenc}
\usepackage[utf8]{inputenc}
\usepackage{graphicx}

\setbeamercovered{dynamic}

\newcommand{\Lang}[1]{\operatorname{\text{\textsc{#1}}}}

\newcommand{\Class}[1]{\operatorname{\mathchoice
  {\text{\sf \small #1}}
  {\text{\sf \small #1}}
  {\text{\sf #1}}
  {\text{\sf #1}}}}

\newcommand{\NumSAT}      {\text{\small\#SAT}}
\newcommand{\NumA}        {\#_{\!A}}

\newcommand{\barA}        {\,\bar{\!A}}

\newcommand{\Nat}{\mathbb{N}}
\newcommand{\Set}[1]{\{#1\}}

\pgfdeclaremask{tu}{beamer-tu-logo-mask}
\pgfdeclaremask{computer}{beamer-computer-mask}
\pgfdeclareimage[interpolate=true,mask=computer,height=2cm]{computerimage}{beamer-computer}
\pgfdeclareimage[interpolate=true,mask=computer,height=2cm]{computerworkingimage}{beamer-computerred}
\pgfdeclareimage[mask=tu,height=.5cm]{logo}{logounesp}

\logo{\pgfuseimage{logo}}

\title{Representação Matricial de Operadores}
\author{Ney Lemke}
\institute[IBB-UNESP]{%
    Mec\^anica Qu\^antica}
\date{2012                                

\colorlet{redshaded}{red!25!bg}
\colorlet{shaded}{black!25!bg}
\colorlet{shadedshaded}{black!10!bg}
\colorlet{blackshaded}{black!40!bg}

\colorlet{darkred}{red!80!black}
\colorlet{darkblue}{blue!80!black}
\colorlet{darkgreen}{green!80!black}

\def\radius{0.96cm}
\def\innerradius{0.85cm}

\def\softness{0.4}
\definecolor{softred}{rgb}{1,\softness,\softness}
\definecolor{softgreen}{rgb}{\softness,1,\softness}
\definecolor{softblue}{rgb}{\softness,\softness,1}

\definecolor{softrg}{rgb}{1,1,\softness}
\definecolor{softrb}{rgb}{1,\softness,1}
\definecolor{softgb}{rgb}{\softness,1,1}

\newcommand{\Bandshaded}[2]{
  \color{shadedshaded}
  \pgfmoveto{\pgfxy(-0.5,0)}
  \pgflineto{\pgfxy(-0.6,0.1)}
  \pgflineto{\pgfxy(-0.4,0.2)}
  \pgflineto{\pgfxy(-0.6,0.3)}
  \pgflineto{\pgfxy(-0.4,0.4)}
  \pgflineto{\pgfxy(-0.5,0.5)}
  \pgflineto{\pgfxy(4,0.5)}
  \pgflineto{\pgfxy(4.1,0.4)}
  \pgflineto{\pgfxy(3.9,0.3)}
  \pgflineto{\pgfxy(4.1,0.2)}
  \pgflineto{\pgfxy(3.9,0.1)}
  \pgflineto{\pgfxy(4,0)}
  \pgfclosepath
  \pgffill

  \color{black}  
  \pgfputat{\pgfxy(0,0.7)}{\pgfbox[left,base]{#1}}
  \pgfputat{\pgfxy(0,-0.1)}{\pgfbox[left,top]{#2}}
}

\newcommand{\Band}[2]{
  \color{shaded}
  \pgfmoveto{\pgfxy(-0.5,0)}
  \pgflineto{\pgfxy(-0.6,0.1)}
  \pgflineto{\pgfxy(-0.4,0.2)}
  \pgflineto{\pgfxy(-0.6,0.3)}
  \pgflineto{\pgfxy(-0.4,0.4)}
  \pgflineto{\pgfxy(-0.5,0.5)}
  \pgflineto{\pgfxy(4,0.5)}
  \pgflineto{\pgfxy(4.1,0.4)}
  \pgflineto{\pgfxy(3.9,0.3)}
  \pgflineto{\pgfxy(4.1,0.2)}
  \pgflineto{\pgfxy(3.9,0.1)}
  \pgflineto{\pgfxy(4,0)}
  \pgfclosepath
  \pgffill

  \color{black}  
  \pgfputat{\pgfxy(0,0.7)}{\pgfbox[left,base]{#1}}
  \pgfputat{\pgfxy(0,-0.1)}{\pgfbox[left,top]{#2}}
}

\newcommand{\BaenderNormal}
{%
  \pgfsetlinewidth{0.4pt}
  \color{black}
  \pgfputat{\pgfxy(0,5)}{\Band{input tapes}{}}
  \pgfputat{\pgfxy(0.35,4.6)}{\pgfbox[center,base]{$\vdots$}}
  \pgfputat{\pgfxy(0,4)}{\Band{}{}}

  \pgfxyline(0,5)(0,5.5)
  \pgfxyline(1.2,5)(1.2,5.5)
  \pgfputat{\pgfxy(0.25,5.25)}{\pgfbox[left,center]{$w_1$}}

  \pgfxyline(0,4)(0,4.5)
  \pgfxyline(1.8,4)(1.8,4.5)        
  \pgfputat{\pgfxy(0.25,4.25)}{\pgfbox[left,center]{$w_n$}}
  \ignorespaces}

\newcommand{\BaenderZweiNormal}
{%
  \pgfsetlinewidth{0.4pt}
  \color{black}
  \pgfputat{\pgfxy(0,5)}{\Band{Zwei Eingabeb\~AƒÂƒ\~A‚¤nder}{}}
  \pgfputat{\pgfxy(0,4.25)}{\Band{}{}}

  \pgfxyline(0,5)(0,5.5)
  \pgfxyline(1.2,5)(1.2,5.5)
  \pgfputat{\pgfxy(0.25,5.25)}{\pgfbox[left,center]{$u$}}

  \pgfxyline(0,4.25)(0,4.75)
  \pgfxyline(1.8,4.25)(1.8,4.75)        
  \pgfputat{\pgfxy(0.25,4.5)}{\pgfbox[left,center]{$v$}}
  \ignorespaces}

\newcommand{\BaenderHell}
{%
  \pgfsetlinewidth{0.4pt}
  \color{black}
  \pgfputat{\pgfxy(0,5)}{\Bandshaded{input tapes}{}}
  \color{shaded}
  \pgfputat{\pgfxy(0.35,4.6)}{\pgfbox[center,base]{$\vdots$}}
  \pgfputat{\pgfxy(0,4)}{\Bandshaded{}{}}

  \color{blackshaded}
  \pgfxyline(0,5)(0,5.5)
  \pgfxyline(1.2,5)(1.2,5.5)
  \pgfputat{\pgfxy(0.25,5.25)}{\pgfbox[left,center]{$w_1$}}

  \pgfxyline(0,4)(0,4.5)
  \pgfxyline(1.8,4)(1.8,4.5)        
  \pgfputat{\pgfxy(0.25,4.25)}{\pgfbox[left,center]{$w_n$}}
  \ignorespaces}

\newcommand{\BaenderZweiHell}
{%
  \pgfsetlinewidth{0.4pt}
  \color{black}
  \pgfputat{\pgfxy(0,5)}{\Bandshaded{Zwei Eingabeb\~AƒÂƒ\~A‚¤nder}{}}%
  \color{blackshaded}
  \pgfputat{\pgfxy(0,4.25)}{\Bandshaded{}{}}
  \pgfputat{\pgfxy(0.25,4.5)}{\pgfbox[left,center]{$v$}}
  \pgfputat{\pgfxy(0.25,5.25)}{\pgfbox[left,center]{$u$}}%

  \pgfxyline(0,5)(0,5.5)
  \pgfxyline(1.2,5)(1.2,5.5)

  \pgfxyline(0,4.25)(0,4.75)
  \pgfxyline(1.8,4.25)(1.8,4.75)        
  \ignorespaces}

\newcommand{\Slot}[1]{%
  \begin{pgftranslate}{\pgfpoint{#1}{0pt}}%
    \pgfsetlinewidth{0.6pt}%
    \color{structure}%
    \pgfmoveto{\pgfxy(-0.1,5.5)}%
    \pgfbezier{\pgfxy(-0.1,5.55)}{\pgfxy(-0.05,5.6)}{\pgfxy(0,5.6)}%
    \pgfbezier{\pgfxy(0.05,5.6)}{\pgfxy(0.1,5.55)}{\pgfxy(0.1,5.5)}%
    \pgflineto{\pgfxy(0.1,4.0)}%
    \pgfbezier{\pgfxy(0.1,3.95)}{\pgfxy(0.05,3.9)}{\pgfxy(0,3.9)}%
    \pgfbezier{\pgfxy(-0.05,3.9)}{\pgfxy(-0.1,3.95)}{\pgfxy(-0.1,4.0)}%
    \pgfclosepath%
    \pgfstroke%
  \end{pgftranslate}\ignorespaces}

\newcommand{\SlotZwei}[1]{%
  \begin{pgftranslate}{\pgfpoint{#1}{0pt}}%
    \pgfsetlinewidth{0.6pt}%
    \color{structure}%
    \pgfmoveto{\pgfxy(-0.1,5.5)}%
    \pgfbezier{\pgfxy(-0.1,5.55)}{\pgfxy(-0.05,5.6)}{\pgfxy(0,5.6)}%
    \pgfbezier{\pgfxy(0.05,5.6)}{\pgfxy(0.1,5.55)}{\pgfxy(0.1,5.5)}%
    \pgflineto{\pgfxy(0.1,4.25)}%
    \pgfbezier{\pgfxy(0.1,4.25)}{\pgfxy(0.05,4.15)}{\pgfxy(0,4.15)}%
    \pgfbezier{\pgfxy(-0.05,4.15)}{\pgfxy(-0.1,4.2)}{\pgfxy(-0.1,4.25)}%
    \pgfclosepath%
    \pgfstroke%
  \end{pgftranslate}\ignorespaces}

\newcommand{\ClipSlot}[1]{%
  \pgfrect[clip]{\pgfrelative{\pgfxy(-0.1,0)}{\pgfpoint{#1}{4cm}}}{\pgfxy(0.2,1.5)}\ignorespaces}

\newcommand{\ClipSlotZwei}[1]{%
  \pgfrect[clip]{\pgfrelative{\pgfxy(-0.1,0)}{\pgfpoint{#1}{4.25cm}}}{\pgfxy(0.2,1.25)}\ignorespaces}


\AtBeginSection[]{\frame{\frametitle{Outline}\tableofcontents[current]}}

\begin{document}
\frame{\titlepage}
\section{Matrizes em Mecânica Quântica}

\frame{\frametitle{Operadores}

$$L^2|lm\rangle =\hbar^2 l(l+1)|lm\rangle $$

$$L_z|lm\rangle=\hbar m |lm\rangle$$

$$|n \rangle =\frac{1}{\sqrt{n!}}(A^\dagger)^n|0\rangle$$

$$H|n\rangle=\hbar\omega \left( n+\frac{1}{2}\right)|n\rangle$$
}

\frame{\frametitle{Operadores}

$$A^\dagger| n\rangle =\sqrt{n+1}|n+1\rangle$$

$$A|n\rangle =\sqrt{n} |n-1\rangle$$

$$\langle m | n\rangle =\delta_{mn}$$

$$I=\sum_{n=0}^\infty |n\rangle\langle n| $$

}

\frame{\frametitle{Sanduíches}

$$\langle m | H| n\rangle =\hbar\omega\left( n+\frac{1}{2} \right)\delta_{nm}$$

$$\langle m |A^\dagger |n\rangle =\sqrt{n+1}\delta_{m,n+1} $$

$$\langle m |A |n\rangle =\sqrt{n}\delta_{m,n-1} $$

}

\frame{\frametitle{Representação Matricial-$P$}


$$
P= \left(
\begin{array}{ccc}
\langle 1 | P | 1 \rangle & \langle 1 | P | 2 \rangle & \cdots \\
\langle 2 | P | 1 \rangle & \langle 2 | P | 2 \rangle & \cdots \\
\vdots & \vdots & \vdots  \\
 \cdots  &  \langle m | P | n\rangle & \cdots \\
\vdots & \vdots & \ddots
\end{array}
\right) 
$$


}

\frame{\frametitle{Hamiltoniano}

$$
H=\hbar\omega \left(
\begin{array}{ccccc}
1/2 & 0 & 0 & 0 & \cdots \\
0  & 3/2 & 0 & 0 & \cdots \\
0  & 0 & 5/2 & 0 & \cdots \\
0  & 0 & 0 & 7/2 & \cdots \\
\vdots & \vdots & \vdots & \vdots & \ddots 
\end{array}
\right) 
$$
}


\frame{\frametitle{$A^\dagger$}
$$
A^\dagger=\left(
\begin{array}{ccccc}
0 & 0 & 0 & 0 & \cdots \\
\sqrt{1}  & 0 & 0 & 0 & \cdots \\
0  & \sqrt{2} & 0 & 0 & \cdots \\
0  & 0 & \sqrt{3} & 0 & \cdots \\
\vdots & \vdots & \vdots & \vdots & \ddots 
\end{array}
\right) 
$$
}

\frame{\frametitle{$A$}
$$
A=\left(
\begin{array}{ccccc}
0 & \sqrt{1} & 0 & 0 & \cdots \\
0  & 0 & \sqrt{2} & 0 & \cdots \\
0  & 0 & 0 & \sqrt{3}  & \cdots \\
0  & 0 & 0 & 0 & \cdots \\
\vdots & \vdots & \vdots & \vdots & \ddots 
\end{array}
\right) 
$$
}


\frame{\frametitle{$L_z$}

$$\langle l^\prime m^\prime |L_z| lm\rangle=
m\hbar\langle l^\prime m^\prime | lm\rangle 
=\hbar m \delta_{mm^\prime}\delta_{ll^\prime}$$


$$
L_z=\hbar \left(
\begin{array}{ccc}
1  & 0 & 0 \\
0  & 0 & 0 \\
0  & 0 & -1 
\end{array}
\right) 
$$
}

\frame{\frametitle{$L_\pm$}

$$\langle l^\prime m^\prime |L_\pm| lm\rangle=
\hbar\sqrt{l(l+1)-m(m\pm 1)}\langle l^\prime m^\prime | lm\pm 1\rangle$$

$$=\hbar\sqrt{l(l+1)-m(m\pm 1)} \delta_{ll^\prime}\delta_{mm^\prime
  \pm 1}$$


$$
L_+=\hbar \left(
\begin{array}{ccc}
0  & \sqrt{2} & 0 \\
0  & 0 & \sqrt{2} \\
0  & 0 & 0 
\end{array}
\right) 
$$
}


\frame{\frametitle{$L_\pm$}
$$
L_-=\hbar \left(
\begin{array}{ccc}
0  & 0  & 0 \\
\sqrt{2}  & 0 & 0 \\
0  & \sqrt{2} & 0 
\end{array}
\right) 
$$

}
\frame{\frametitle{$L_x$}
$$
L_x=\frac{L_++L_-}{2}=\frac{\hbar}{\sqrt{2}} 
\left(
\begin{array}{ccc}
0  & 1 & 0 \\
1  & 0 & 1 \\
0  & 1 & 0 
\end{array}
\right) 
$$
}

\frame{\frametitle{$L_y$}
$$L_y=\frac{-i(L_+-L_-)}{2}=\frac{\hbar}{\sqrt{2}} 
\left(
\begin{array}{ccc}
0  & -i & 0 \\
i  & 0 & -i \\
0  & i  & 0 
\end{array}
\right) 
$$
}

\frame{\frametitle{Exemplo}
Calcule usando matrizes $[L_x,L_y]=i\hbar L_z$


}

\frame{\frametitle{Bras e Kets}

Em uma base $|n\rangle$ podemos expressar um ket $|\psi\rangle$

$$
\langle n |\psi\rangle\rightarrow \left(
\begin{array}{c}
\langle 1 | \psi \rangle \\
\langle 2 | \psi \rangle \\
\langle 3 | \psi \rangle \\
\vdots
\end{array}
\right)
\rightarrow \left(
\begin{array}{c}
\alpha_1  \\
\alpha_2 \\
\alpha_3 \\
\vdots
\end{array}
\right)
$$
}

\frame{\frametitle{Bras e Kets}

De forma análoga:

$$
\langle n|\phi\rangle\rightarrow \left(
\begin{array}{c}
\langle 1 | \phi \rangle \\
\langle 2 | \phi \rangle \\
\langle 3 | \phi \rangle \\
\vdots
\end{array}
\right)
\rightarrow \left(
\begin{array}{c}
\beta_1  \\
\beta_2 \\
\beta_3 \\
\vdots
\end{array}
\right)
$$
}


\frame{\frametitle{Bras e Kets}
$$\langle \psi |n\rangle \rightarrow (\alpha_1^*,\alpha_2^*,\cdots )$$

$$\langle \psi | \phi \rangle =
\sum_n \langle \psi | n\rangle \langle n | \phi \rangle=
(\alpha_1^*,\alpha_2^*,\cdots )
 \left(
\begin{array}{c}
\beta_1  \\
\beta_2 \\
\beta_3 \\
\vdots
\end{array}
\right)=\sum_n\beta_n\alpha_n^*
$$

}

\frame{\frametitle{Sanduíches}

$$\langle \phi | Q| \psi \rangle =
\sum_n \langle \psi | n\rangle \langle n | \phi \rangle=
(\alpha_1^*,\alpha_2^*,\cdots )
 \left(
\begin{array}{ccc}
Q_{11} & Q_{12}  & \cdots \\
Q_{21} & Q_{22} & \cdots \\
\vdots & \vdots & \vdots  \\
\vdots & \vdots & \ddots
\end{array}
\right)
 \left(
\begin{array}{c}
\beta_1  \\
\beta_2 \\
\beta_3 \\
\vdots
\end{array}
\right)$$

$$=\sum_{nm}\alpha_n^* Q_{nm}\beta_m^*$$
}

\frame{\frametitle{Exercícios}

Obtenha a representação matricial dos seguintes operadores:

\begin{itemize}
\item $x$
\item $x^2$
\item $p$
\item $px$
\item $xp$
\end{itemize}
}
\end{document}
