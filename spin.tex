% $Header: /cvsroot/latex-beamer/latex-beamer/examples/beamerexample5.tex,v 1.22 2004/10/08 14:02:33 tantau Exp $

\documentclass[11pt]{beamer}

\usetheme{Darmstadt}

\usepackage{times}
\usefonttheme{structurebold}

%\usepackage[english]{babel}
\usepackage[portuges]{babel}
\usepackage{pgf,pgfarrows,pgfnodes,pgfautomata,pgfheaps}
\usepackage{amsmath,amssymb}
%\usepackage[latin8]{inputenc}
\usepackage[utf8]{inputenc}
\usepackage{graphicx}

\setbeamercovered{dynamic}

\newcommand{\Lang}[1]{\operatorname{\text{\textsc{#1}}}}

\newcommand{\Class}[1]{\operatorname{\mathchoice
  {\text{\sf \small #1}}
  {\text{\sf \small #1}}
  {\text{\sf #1}}
  {\text{\sf #1}}}}

\newcommand{\NumSAT}      {\text{\small\#SAT}}
\newcommand{\NumA}        {\#_{\!A}}

\newcommand{\barA}        {\,\bar{\!A}}

\newcommand{\Nat}{\mathbb{N}}
\newcommand{\Set}[1]{\{#1\}}

\pgfdeclaremask{tu}{beamer-tu-logo-mask}
\pgfdeclaremask{computer}{beamer-computer-mask}
\pgfdeclareimage[interpolate=true,mask=computer,height=2cm]{computerimage}{beamer-computer}
\pgfdeclareimage[interpolate=true,mask=computer,height=2cm]{computerworkingimage}{beamer-computerred}
\pgfdeclareimage[mask=tu,height=.5cm]{logo}{logounesp}

\logo{\pgfuseimage{logo}}

\title{Spin}
\author{Ney Lemke}
\institute[IBB-UNESP]{%
    Mec\^anica Qu\^antica}
\date{2012}                                

\colorlet{redshaded}{red!25!bg}
\colorlet{shaded}{black!25!bg}
\colorlet{shadedshaded}{black!10!bg}
\colorlet{blackshaded}{black!40!bg}

\colorlet{darkred}{red!80!black}
\colorlet{darkblue}{blue!80!black}
\colorlet{darkgreen}{green!80!black}

\def\radius{0.96cm}
\def\innerradius{0.85cm}

\def\softness{0.4}
\definecolor{softred}{rgb}{1,\softness,\softness}
\definecolor{softgreen}{rgb}{\softness,1,\softness}
\definecolor{softblue}{rgb}{\softness,\softness,1}

\definecolor{softrg}{rgb}{1,1,\softness}
\definecolor{softrb}{rgb}{1,\softness,1}
\definecolor{softgb}{rgb}{\softness,1,1}

\newcommand{\Bandshaded}[2]{
  \color{shadedshaded}
  \pgfmoveto{\pgfxy(-0.5,0)}
  \pgflineto{\pgfxy(-0.6,0.1)}
  \pgflineto{\pgfxy(-0.4,0.2)}
  \pgflineto{\pgfxy(-0.6,0.3)}
  \pgflineto{\pgfxy(-0.4,0.4)}
  \pgflineto{\pgfxy(-0.5,0.5)}
  \pgflineto{\pgfxy(4,0.5)}
  \pgflineto{\pgfxy(4.1,0.4)}
  \pgflineto{\pgfxy(3.9,0.3)}
  \pgflineto{\pgfxy(4.1,0.2)}
  \pgflineto{\pgfxy(3.9,0.1)}
  \pgflineto{\pgfxy(4,0)}
  \pgfclosepath
  \pgffill

  \color{black}  
  \pgfputat{\pgfxy(0,0.7)}{\pgfbox[left,base]{#1}}
  \pgfputat{\pgfxy(0,-0.1)}{\pgfbox[left,top]{#2}}
}

\newcommand{\Band}[2]{
  \color{shaded}
  \pgfmoveto{\pgfxy(-0.5,0)}
  \pgflineto{\pgfxy(-0.6,0.1)}
  \pgflineto{\pgfxy(-0.4,0.2)}
  \pgflineto{\pgfxy(-0.6,0.3)}
  \pgflineto{\pgfxy(-0.4,0.4)}
  \pgflineto{\pgfxy(-0.5,0.5)}
  \pgflineto{\pgfxy(4,0.5)}
  \pgflineto{\pgfxy(4.1,0.4)}
  \pgflineto{\pgfxy(3.9,0.3)}
  \pgflineto{\pgfxy(4.1,0.2)}
  \pgflineto{\pgfxy(3.9,0.1)}
  \pgflineto{\pgfxy(4,0)}
  \pgfclosepath
  \pgffill

  \color{black}  
  \pgfputat{\pgfxy(0,0.7)}{\pgfbox[left,base]{#1}}
  \pgfputat{\pgfxy(0,-0.1)}{\pgfbox[left,top]{#2}}
}

\newcommand{\BaenderNormal}
{%
  \pgfsetlinewidth{0.4pt}
  \color{black}
  \pgfputat{\pgfxy(0,5)}{\Band{input tapes}{}}
  \pgfputat{\pgfxy(0.35,4.6)}{\pgfbox[center,base]{$\vdots$}}
  \pgfputat{\pgfxy(0,4)}{\Band{}{}}

  \pgfxyline(0,5)(0,5.5)
  \pgfxyline(1.2,5)(1.2,5.5)
  \pgfputat{\pgfxy(0.25,5.25)}{\pgfbox[left,center]{$w_1$}}

  \pgfxyline(0,4)(0,4.5)
  \pgfxyline(1.8,4)(1.8,4.5)        
  \pgfputat{\pgfxy(0.25,4.25)}{\pgfbox[left,center]{$w_n$}}
  \ignorespaces}

\newcommand{\BaenderZweiNormal}
{%
  \pgfsetlinewidth{0.4pt}
  \color{black}
  \pgfputat{\pgfxy(0,5)}{\Band{Zwei Eingabeb\~AƒÂƒ\~A‚¤nder}{}}
  \pgfputat{\pgfxy(0,4.25)}{\Band{}{}}

  \pgfxyline(0,5)(0,5.5)
  \pgfxyline(1.2,5)(1.2,5.5)
  \pgfputat{\pgfxy(0.25,5.25)}{\pgfbox[left,center]{$u$}}

  \pgfxyline(0,4.25)(0,4.75)
  \pgfxyline(1.8,4.25)(1.8,4.75)        
  \pgfputat{\pgfxy(0.25,4.5)}{\pgfbox[left,center]{$v$}}
  \ignorespaces}

\newcommand{\BaenderHell}
{%
  \pgfsetlinewidth{0.4pt}
  \color{black}
  \pgfputat{\pgfxy(0,5)}{\Bandshaded{input tapes}{}}
  \color{shaded}
  \pgfputat{\pgfxy(0.35,4.6)}{\pgfbox[center,base]{$\vdots$}}
  \pgfputat{\pgfxy(0,4)}{\Bandshaded{}{}}

  \color{blackshaded}
  \pgfxyline(0,5)(0,5.5)
  \pgfxyline(1.2,5)(1.2,5.5)
  \pgfputat{\pgfxy(0.25,5.25)}{\pgfbox[left,center]{$w_1$}}

  \pgfxyline(0,4)(0,4.5)
  \pgfxyline(1.8,4)(1.8,4.5)        
  \pgfputat{\pgfxy(0.25,4.25)}{\pgfbox[left,center]{$w_n$}}
  \ignorespaces}

\newcommand{\BaenderZweiHell}
{%
  \pgfsetlinewidth{0.4pt}
  \color{black}
  \pgfputat{\pgfxy(0,5)}{\Bandshaded{Zwei Eingabeb\~AƒÂƒ\~A‚¤nder}{}}%
  \color{blackshaded}
  \pgfputat{\pgfxy(0,4.25)}{\Bandshaded{}{}}
  \pgfputat{\pgfxy(0.25,4.5)}{\pgfbox[left,center]{$v$}}
  \pgfputat{\pgfxy(0.25,5.25)}{\pgfbox[left,center]{$u$}}%

  \pgfxyline(0,5)(0,5.5)
  \pgfxyline(1.2,5)(1.2,5.5)

  \pgfxyline(0,4.25)(0,4.75)
  \pgfxyline(1.8,4.25)(1.8,4.75)        
  \ignorespaces}

\newcommand{\Slot}[1]{%
  \begin{pgftranslate}{\pgfpoint{#1}{0pt}}%
    \pgfsetlinewidth{0.6pt}%
    \color{structure}%
    \pgfmoveto{\pgfxy(-0.1,5.5)}%
    \pgfbezier{\pgfxy(-0.1,5.55)}{\pgfxy(-0.05,5.6)}{\pgfxy(0,5.6)}%
    \pgfbezier{\pgfxy(0.05,5.6)}{\pgfxy(0.1,5.55)}{\pgfxy(0.1,5.5)}%
    \pgflineto{\pgfxy(0.1,4.0)}%
    \pgfbezier{\pgfxy(0.1,3.95)}{\pgfxy(0.05,3.9)}{\pgfxy(0,3.9)}%
    \pgfbezier{\pgfxy(-0.05,3.9)}{\pgfxy(-0.1,3.95)}{\pgfxy(-0.1,4.0)}%
    \pgfclosepath%
    \pgfstroke%
  \end{pgftranslate}\ignorespaces}

\newcommand{\SlotZwei}[1]{%
  \begin{pgftranslate}{\pgfpoint{#1}{0pt}}%
    \pgfsetlinewidth{0.6pt}%
    \color{structure}%
    \pgfmoveto{\pgfxy(-0.1,5.5)}%
    \pgfbezier{\pgfxy(-0.1,5.55)}{\pgfxy(-0.05,5.6)}{\pgfxy(0,5.6)}%
    \pgfbezier{\pgfxy(0.05,5.6)}{\pgfxy(0.1,5.55)}{\pgfxy(0.1,5.5)}%
    \pgflineto{\pgfxy(0.1,4.25)}%
    \pgfbezier{\pgfxy(0.1,4.25)}{\pgfxy(0.05,4.15)}{\pgfxy(0,4.15)}%
    \pgfbezier{\pgfxy(-0.05,4.15)}{\pgfxy(-0.1,4.2)}{\pgfxy(-0.1,4.25)}%
    \pgfclosepath%
    \pgfstroke%
  \end{pgftranslate}\ignorespaces}

\newcommand{\ClipSlot}[1]{%
  \pgfrect[clip]{\pgfrelative{\pgfxy(-0.1,0)}{\pgfpoint{#1}{4cm}}}{\pgfxy(0.2,1.5)}\ignorespaces}

\newcommand{\ClipSlotZwei}[1]{%
  \pgfrect[clip]{\pgfrelative{\pgfxy(-0.1,0)}{\pgfpoint{#1}{4.25cm}}}{\pgfxy(0.2,1.25)}\ignorespaces}


\AtBeginSection[]{\frame{\frametitle{Outline}\tableofcontents[current]}}

\begin{document}
\frame{\titlepage}
\frame{\frametitle{Operadores} 

$$[S_i,S_j]=\epsilon_{ijk}S_k$$

$$\langle l^\prime m^\prime |L_\pm| lm\rangle=\hbar\sqrt{l(l+1)-m(m\pm 1)} \delta_{ll^\prime}\delta_{mm\pm 1}$$

$$L_x=\frac{L_++L_-}{2}$$
$$L_y=\frac{-i(L_+-L_-)}{2}$$
}

\frame{\frametitle{Matrizes} 
$$S_z=\hbar\left( 
\begin{array}{cc}
1/2 & 0 \\
0   & -1/2 \\
\end{array}
\right)
$$

$$S_+=\hbar\left( 
\begin{array}{cc}
0 & 1 \\
0   & 0 \\
\end{array}
\right)
\quad 
S_-=\hbar\left( 
\begin{array}{cc}
0 & 0 \\
1 & 0 \\
\end{array}
\right)
\quad 
$$
}

\frame{\frametitle{Matrizes de Pauli} 
$$\vec{S}=\frac{\hbar}{2}\vec{\sigma}$$

$$I=\left( 
\begin{array}{cc}
1 & 0 \\
0  & 1 \\
\end{array}
\right)
\quad 
\sigma_x=\left( 
\begin{array}{cc}
0 & 1 \\
1 & 0 \\
\end{array}
\right)
$$

$$\sigma_y=\left( 
\begin{array}{cc}
0 & -i \\
i  & 0 \\
\end{array}
\right)
\quad 
\sigma_z=\left( 
\begin{array}{cc}
1 & 0 \\
0 & -1 \\
\end{array}
\right)
$$
}


\frame{\frametitle{Propriedades} 
  \begin{itemize}
  \item $$[\sigma_i,\sigma_j]=2\epsilon_{ijk}\sigma_z$$
  \item $$\sigma_x^2=\sigma_y^2=\sigma_z^2=I$$
  \item $$\{\sigma_i, \sigma_j\}=0$$
  \end{itemize}
Nota $\{A,B\}=AB+BA$ chamado de anticomutador. 
}

\frame{\frametitle{Spinores} 
 
$$ S_z\chi_\pm =\pm \hbar \chi_\pm$$

$\chi_\pm$ são chamados de spinores.

Em termos matriciais:

 $$\hbar\left( 
 \begin{array}{cc}
 1/2 & 0 \\
 0   & -1/2 \\
 \end{array}
 \right)
\left( 
   \begin{array}{c}
     u \\ 
     v
   \end{array}
 \right)
=\frac{\pm \hbar}{2}
 \left( 
   \begin{array}{c}
     u \\ 
     v
   \end{array}
 \right)
$$
}

\frame{\frametitle{Spinores} 
Qualquer spinor pode ser escrito como:

$$\chi=\alpha_+\left( 
  \begin{array}{c}
    1 \\ 
    0
  \end{array}
\right)+
\alpha_-\left( 
  \begin{array}{c}
    0 \\ 
    1
  \end{array}
\right)=
\left( 
  \begin{array}{c}
    \alpha_+ \\ 
    \alpha_-
  \end{array}
\right)
$$
}
\frame{\frametitle{Caso Geral} 
$$S=S_x\cos \phi +S_y \sin \phi$$

$$S\chi =\frac{\hbar}{2} \lambda \chi $$
Em termos matriciais:

$$
\frac{\hbar}{2}\left( 
 \begin{array}{cc}
 0 & \cos \phi-i\sin \phi \\
 \cos \phi +i \sin \phi    & 0 \\
 \end{array}
 \right)
\left( 
   \begin{array}{c}
     u \\ 
     v
   \end{array}
 \right)
=\frac{\hbar\lambda}{2}
 \left( 
   \begin{array}{c}
     u \\ 
     v
   \end{array}
 \right)
$$
}
\frame{\frametitle{Caso Geral-Solução} 

$$\lambda_\pm=\pm 1$$

$$\chi_+=\frac{1}{\sqrt{2}}
 \left( 
   \begin{array}{c}
     e^{-i\phi/2} \\ 
     e^{i\phi/2}
   \end{array}
 \right)
\quad
\chi_-=\frac{1}{\sqrt{2}}
 \left( 
   \begin{array}{c}
     e^{i\phi/2} \\ 
     -e^{-i\phi/2}
   \end{array}
 \right)
$$
}

\frame{\frametitle{Valores Esperados} 

$$\langle \alpha| \vec{S} |\alpha\rangle =\sum_{ij} 
\langle \alpha|i\rangle \langle i |\vec{S}|j\rangle\langle j | \alpha \rangle
$$

Na forma vetorial:
$$\langle \alpha| \vec{S} |\alpha\rangle=
(\alpha_+^*,\alpha_-^*).\vec{S}.
\left( 
   \begin{array}{c}
     \alpha_+ \\ 
     \alpha_- 
   \end{array}
 \right)
$$
}

\frame{\frametitle{Valores Esperados} 
$$\langle S_x \rangle = 
\langle \alpha|S_x|\alpha\rangle=
(\alpha_+^*,\alpha_-^*).
\frac{\hbar}{2}
\left( 
\begin{array}{cc}
0 & 1 \\
1 & 0 \\
\end{array}
\right).\left( 
   \begin{array}{c}
     \alpha_+ \\ 
     \alpha_- 
   \end{array}
 \right)
$$

$$\langle S_x \rangle =\frac{\hbar}{2}(\alpha_+^*\alpha_-+\alpha_-^*\alpha_+)$$


}

\frame{\frametitle{Valores Esperados} 
$$\langle S_y \rangle = 
\langle \alpha|S_y|\alpha\rangle=
(\alpha_+^*,\alpha_-^*).
\frac{\hbar}{2}
\left( 
\begin{array}{cc}
0 & -i \\
i & 0 \\
\end{array}
\right).\left( 
   \begin{array}{c}
     \alpha_+ \\ 
     \alpha_- 
   \end{array}
 \right)
$$

$$\langle S_y \rangle =\frac{i\hbar}{2}(\alpha_+^*\alpha_--\alpha_-^*\alpha_+)$$


}


\frame{\frametitle{Valores Esperados} 
$$\langle S_z \rangle = 
\langle \alpha|S_y|\alpha\rangle=
(\alpha_+^*,\alpha_-^*).
\frac{\hbar}{2}
\left( 
\begin{array}{cc}
1 & 0 \\
0 & 1 \\
\end{array}
\right).
\left( 
   \begin{array}{c}
     \alpha_+ \\ 
     \alpha_- 
   \end{array}
 \right)
$$

$$\langle S_z \rangle =\frac{\hbar}{2}(\alpha_+^*\alpha_+-\alpha_-^*\alpha_-)
=\frac{\hbar}{2}(|\alpha_+|^2-|\alpha_-|^2)$$


}

\frame{\frametitle{Exemplo} 
Uma medida de $S_x$ obtém o valor $\frac{\hbar}{2}$. Uma medida subsequente de $S_x\cos \phi +S_y \sin \phi$ é realizada qual é a probabilidade de se obter
$\hbar/2$?

}

\frame{\frametitle{Resposta}

$P_+=\cos^2 \frac{\phi}{2} $
} 

\frame{\frametitle{Momento Magnético}

$$\vec{M}=\frac{-eg}{2m_e}\vec{S}$$

onde $g=2\times (1,0011596)$
}

\frame{\frametitle{Momento Magnético}
Classicamente:

$$\vec{M}=-\frac{e\vec{L}}{2m}$$

A corrente $I$ é dada por $e/\tau$ onde $\tau$ é o período. 

$$L=rp=m_erv \quad v=\frac{2\pi r}{\tau}$$

$$L=\frac{2\pi m_er^2}{\tau}$$
}

\frame{\frametitle{Momento Magnético}

O momento magnético é dado por 

$$M=IS$$

onde $S$ é a área do circuito.

$$M=\frac{e}{\tau}\pi r^2$$

Usando o valor de $L$ temos que:

$$M=\frac{eL}{2m_e}$$
}

\frame{\frametitle{Spin em um Campo  Magnético}
$$H=-\vec{M}.\vec{B}=\frac{eg\hbar}{4m_e}\vec{\sigma}.\vec{B}$$

$$i\hbar \frac{d}{dt}|\psi(t)\rangle = 
\frac{eg\hbar}{4m_e}\vec{\sigma}.\vec{B}|\psi(t)\rangle$$

$$\vec{B}=B_o\vec{a_z}$$ 

$$|\psi(t)\rangle=e^{i\omega t}\left( 
   \begin{array}{c}
     \alpha_+^* \\ 
     \alpha_-^* 
   \end{array}
 \right)
$$

$$
|\psi(t)\rangle =
\left( 
   \begin{array}{c}
     ae^{i\omega_o t} \\ 
     be^{-i\omega_o t} 
   \end{array}
 \right)$$
}

\frame{\frametitle{Exemplo}
Em $t=0$ $\psi$ é o autoestado de $S_x$ com autovalor $\hbar/2$. 

$$|\psi(t)\rangle=\frac{1}{\sqrt{2}}\left( 
   \begin{array}{c}
     e^{i\omega_o t} \\ 
     e^{-i\omega_o t} 
   \end{array}
 \right)$$

$$\langle S_x\rangle =\frac{\hbar}{2} \cos 2\omega_o t$$
$$\langle S_y\rangle =\frac{\hbar}{2} \sin 2\omega_o t$$

$$2\omega_o=\frac{egB}{2m_e}$$

$$B=1T$$ $$\omega_c=0.9\times 10^{11} rad/s$$


}

\frame{\frametitle{Ressonância}

Em sólidos $g$ depende do campo na proximidade dos núcleos. 

Vamos supor que temos um campo aplicado na direção $z$,
$B_o$,  e um campo perpendicular
$B_x=B_1 \cos \omega t$. 

A eq. de Schrödinger neste caso é:

 $$i\hbar \frac{d}{dt} 
 \left( \begin{array}{c}
 a(t) \\
 b(t)
 \end{array}
 \right) 
 =
 \frac{eg\hbar}{4 m_e}
 \left( \begin{array}{cc}
B_o & B_1\cos \omega t \\
B_1\cos \omega t  & -B_o 
\end{array}
\right)
\left( \begin{array}{c}
a(t) \\
b(t)
\end{array}
\right) 
 $$
}


\frame{\frametitle{Anz\"atz}

$$A(t)=a(t) e^{i\omega_o t} \quad B(t)=b(t)e^{i\omega_o t}$$

$$\omega_1=\frac{egB_1}{4 m_e} \quad \omega_o=\frac{egB_o}{4 m_e}$$

$$i\frac{dA(t)}{dt}=-\omega_o a(t) +\frac{da(t)}{dt}e^{i\omega_o t}$$

$$=-\omega_o a(t) +(\omega_o a(t) +\omega_1\cos \omega t )e^{i\omega_ot}$$

$$=\omega_1B(t)e^{2i\omega_o t}\cos \omega t$$ 

}

\frame{\frametitle{Anz\"atz}
$$i\frac{dB(t)}{dt}=\omega_1A(t)e^{-2i\omega_o t}\cos \omega t$$ 

Desprezamos os termos que oscilam rápido, pois vamos estar interessados 
em valores médios. 

$$i\frac{dA(t)}{dt}=\frac{\omega_1}{2} A(t)e^{2i\omega_o t-i\omega t}$$

$$i\frac{dB(t)}{dt}=\frac{\omega_1}{2} B(t)e^{-(2i\omega_o t-\omega t)}$$

$$\frac{d^2A(t)}{dt^2}=i(2\omega_o -\omega)\frac{dA}{dt}-\frac{\omega_1^2}{4}A(t)$$

$$A(t)=A(0)e^{i\Omega t}$$


}

\frame{\frametitle{Solução}


$$\Omega_\pm =\left(\omega_o \pm \frac{\omega}{2} \right)\pm \sqrt{
\left(\omega_o -\frac{\omega}{2} \right)^2+\frac{\omega_1^2}{4}}
$$


$$A(t)=A_+ e^{i\Omega_+ t}+A_- e^{i\Omega_- t}$$

$$B(t)=\frac{-2}{\omega_1}\left(
A_+ \Omega_+ e^{i\Omega_+ t}+A_-\Omega_- e^{i\Omega_- t}
\right)
$$
}

\frame{\frametitle{Solução}
Suponha que em $t=0$ $a(0)=1$ e $b(0)=0$. 
Lembrando que:

$$A(t)=a(t) e^{i\omega_o t} \quad B(t)=b(t)e^{i\omega_o t}$$

$$A_++A_-=1 \quad \Omega_+A_++\Omega_-A_-=0$$

Resolvendo temos que:

$$A_+=\frac{\Omega_-}{\Omega_--\Omega_+} \quad
A_-=\frac{-\Omega_+}{\Omega_--\Omega_+} $$
}

\frame{\frametitle{Solução}
Qual é a probabilidade de ficar na posição reversa?


$$P_-(t)=|b(t)|^2=\frac{4}{\omega_1^2}\left|\frac{\Omega_+\Omega_-}{\Omega_+ -\Omega_-} \right|^2(2-2 \cos (\Omega_+ -\Omega_-)t)$$


Na ressonância:

$$\omega=2\omega_o$$

$$\Omega_\pm=\frac{\pm \omega_1}{2}$$

$$P_{res}=\frac{1}{2}(1-\cos \omega_1 t)$$
}

\frame{\frametitle{Solução}
Fora da ressonância:

$$P_-(t)=\frac{1}{2}\frac{\omega_1^2}{(2\omega_o-\omega)^2+\omega_1^2}
(1-\cos \sqrt{(2\omega_o -\omega)^2+\omega_1^2}t)$$
}
\end{document}



