\thispagestyle{headandfoot}
\begin{center} {\large Verifica��o da RER}
\end{center}
\vspace{0.5cm} Nome:\rule{14cm}{0.01cm} \\


\vspace{1 cm}
 
{\bf S\'o ser\~ao aceitas respostas devidamente justificadas.}

\vspace{1 cm}
\begin{questions}

\question[2.5]   Use as regras de quantiza��o de Bohr para determinar os 
n�veis de energia de uma part�cula sob a a��o de um potencial 
$V(r)=\alpha r$. Assuma �rbitas circulares.

\question[2.5] Considere a fun��o de onda na representa��o de momento 
$$\phi(p)=Ne^{-\alpha p^2+i\beta}$$, $\alpha$ e $\beta$ reais  e
determine:
\begin{itemize}
\item Determine $N$ para que $\phi$ seja normalizada.
\item A fun��o de onda no espa�o de posi��o.
\item O valor esperado da posi��o (n�o � preciso calcular a integral)
\item A posi��o com maior probabilidade de se encontrar a part�cula.
\end{itemize}
 
\question[1.5] Calcule $\langle l,m_1|L_x L_y| l,m_2 \rangle$. 

\question[1.5] Calcule $[L_y,x]$.

\question[2.0] Considere um sistema de spin 1/2, voc� realiza uma medida
de $S_x+2 S_y+2S_z$ e obt�m o valor $3 \hbar/2$. Qual � a probabilidade de  
se obter em uma  medida subsequente de $S_y$ $\hbar/2$? 
 
\end{questions}





%%% Local Variables: 
%%% mode: latex
%%% TeX-master: "exame"
%%% TeX-master: "exame"
%%% TeX-master: "exame"
%%% TeX-master: "exame"
%%% End: 
