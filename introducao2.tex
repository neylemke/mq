% $Header: /cvsroot/latex-beamer/latex-beamer/examples/beamerexample5.tex,v 1.22 2004/10/08 14:02:33 tantau Exp $

\documentclass[11pt]{beamer}

\usetheme{Darmstadt}

\usepackage{times}
\usefonttheme{structurebold}

%\usepackage[english]{babel}
\usepackage[portuges]{babel}
\usepackage{pgf,pgfarrows,pgfnodes,pgfautomata,pgfheaps}
\usepackage{amsmath,amssymb}
\usepackage[latin1]{inputenc}
\usepackage{graphicx}

\setbeamercovered{dynamic}

\newcommand{\Lang}[1]{\operatorname{\text{\textsc{#1}}}}

\newcommand{\Class}[1]{\operatorname{\mathchoice
  {\text{\sf \small #1}}
  {\text{\sf \small #1}}
  {\text{\sf #1}}
  {\text{\sf #1}}}}

\newcommand{\NumSAT}      {\text{\small\#SAT}}
\newcommand{\NumA}        {\#_{\!A}}

\newcommand{\barA}        {\,\bar{\!A}}

\newcommand{\Nat}{\mathbb{N}}
\newcommand{\Set}[1]{\{#1\}}

\pgfdeclaremask{tu}{beamer-tu-logo-mask}
\pgfdeclaremask{computer}{beamer-computer-mask}
\pgfdeclareimage[interpolate=true,mask=computer,height=2cm]{computerimage}{beamer-computer}
\pgfdeclareimage[interpolate=true,mask=computer,height=2cm]{computerworkingimage}{beamer-computerred}
\pgfdeclareimage[mask=tu,height=.5cm]{logo}{logounesp}

\logo{\pgfuseimage{logo}}

\title{Introduç±ao}
\author{Ney Lemke}
\institute[IBB-UNESP]{%
    Mecânica Quântica \\
    Departamento de Física e Biofísica}
\date{ 2008}                                

\colorlet{redshaded}{red!25!bg}
\colorlet{shaded}{black!25!bg}
\colorlet{shadedshaded}{black!10!bg}
\colorlet{blackshaded}{black!40!bg}

\colorlet{darkred}{red!80!black}
\colorlet{darkblue}{blue!80!black}
\colorlet{darkgreen}{green!80!black}

\def\radius{0.96cm}
\def\innerradius{0.85cm}

\def\softness{0.4}
\definecolor{softred}{rgb}{1,\softness,\softness}
\definecolor{softgreen}{rgb}{\softness,1,\softness}
\definecolor{softblue}{rgb}{\softness,\softness,1}

\definecolor{softrg}{rgb}{1,1,\softness}
\definecolor{softrb}{rgb}{1,\softness,1}
\definecolor{softgb}{rgb}{\softness,1,1}

\newcommand{\Bandshaded}[2]{
  \color{shadedshaded}
  \pgfmoveto{\pgfxy(-0.5,0)}
  \pgflineto{\pgfxy(-0.6,0.1)}
  \pgflineto{\pgfxy(-0.4,0.2)}
  \pgflineto{\pgfxy(-0.6,0.3)}
  \pgflineto{\pgfxy(-0.4,0.4)}
  \pgflineto{\pgfxy(-0.5,0.5)}
  \pgflineto{\pgfxy(4,0.5)}
  \pgflineto{\pgfxy(4.1,0.4)}
  \pgflineto{\pgfxy(3.9,0.3)}
  \pgflineto{\pgfxy(4.1,0.2)}
  \pgflineto{\pgfxy(3.9,0.1)}
  \pgflineto{\pgfxy(4,0)}
  \pgfclosepath
  \pgffill

  \color{black}  
  \pgfputat{\pgfxy(0,0.7)}{\pgfbox[left,base]{#1}}
  \pgfputat{\pgfxy(0,-0.1)}{\pgfbox[left,top]{#2}}
}

\newcommand{\Band}[2]{
  \color{shaded}
  \pgfmoveto{\pgfxy(-0.5,0)}
  \pgflineto{\pgfxy(-0.6,0.1)}
  \pgflineto{\pgfxy(-0.4,0.2)}
  \pgflineto{\pgfxy(-0.6,0.3)}
  \pgflineto{\pgfxy(-0.4,0.4)}
  \pgflineto{\pgfxy(-0.5,0.5)}
  \pgflineto{\pgfxy(4,0.5)}
  \pgflineto{\pgfxy(4.1,0.4)}
  \pgflineto{\pgfxy(3.9,0.3)}
  \pgflineto{\pgfxy(4.1,0.2)}
  \pgflineto{\pgfxy(3.9,0.1)}
  \pgflineto{\pgfxy(4,0)}
  \pgfclosepath
  \pgffill

  \color{black}  
  \pgfputat{\pgfxy(0,0.7)}{\pgfbox[left,base]{#1}}
  \pgfputat{\pgfxy(0,-0.1)}{\pgfbox[left,top]{#2}}
}

\newcommand{\BaenderNormal}
{%
  \pgfsetlinewidth{0.4pt}
  \color{black}
  \pgfputat{\pgfxy(0,5)}{\Band{input tapes}{}}
  \pgfputat{\pgfxy(0.35,4.6)}{\pgfbox[center,base]{$\vdots$}}
  \pgfputat{\pgfxy(0,4)}{\Band{}{}}

  \pgfxyline(0,5)(0,5.5)
  \pgfxyline(1.2,5)(1.2,5.5)
  \pgfputat{\pgfxy(0.25,5.25)}{\pgfbox[left,center]{$w_1$}}

  \pgfxyline(0,4)(0,4.5)
  \pgfxyline(1.8,4)(1.8,4.5)        
  \pgfputat{\pgfxy(0.25,4.25)}{\pgfbox[left,center]{$w_n$}}
  \ignorespaces}

\newcommand{\BaenderZweiNormal}
{%
  \pgfsetlinewidth{0.4pt}
  \color{black}
  \pgfputat{\pgfxy(0,5)}{\Band{Zwei Eingabebänder}{}}
  \pgfputat{\pgfxy(0,4.25)}{\Band{}{}}

  \pgfxyline(0,5)(0,5.5)
  \pgfxyline(1.2,5)(1.2,5.5)
  \pgfputat{\pgfxy(0.25,5.25)}{\pgfbox[left,center]{$u$}}

  \pgfxyline(0,4.25)(0,4.75)
  \pgfxyline(1.8,4.25)(1.8,4.75)        
  \pgfputat{\pgfxy(0.25,4.5)}{\pgfbox[left,center]{$v$}}
  \ignorespaces}

\newcommand{\BaenderHell}
{%
  \pgfsetlinewidth{0.4pt}
  \color{black}
  \pgfputat{\pgfxy(0,5)}{\Bandshaded{input tapes}{}}
  \color{shaded}
  \pgfputat{\pgfxy(0.35,4.6)}{\pgfbox[center,base]{$\vdots$}}
  \pgfputat{\pgfxy(0,4)}{\Bandshaded{}{}}

  \color{blackshaded}
  \pgfxyline(0,5)(0,5.5)
  \pgfxyline(1.2,5)(1.2,5.5)
  \pgfputat{\pgfxy(0.25,5.25)}{\pgfbox[left,center]{$w_1$}}

  \pgfxyline(0,4)(0,4.5)
  \pgfxyline(1.8,4)(1.8,4.5)        
  \pgfputat{\pgfxy(0.25,4.25)}{\pgfbox[left,center]{$w_n$}}
  \ignorespaces}

\newcommand{\BaenderZweiHell}
{%
  \pgfsetlinewidth{0.4pt}
  \color{black}
  \pgfputat{\pgfxy(0,5)}{\Bandshaded{Zwei Eingabebänder}{}}%
  \color{blackshaded}
  \pgfputat{\pgfxy(0,4.25)}{\Bandshaded{}{}}
  \pgfputat{\pgfxy(0.25,4.5)}{\pgfbox[left,center]{$v$}}
  \pgfputat{\pgfxy(0.25,5.25)}{\pgfbox[left,center]{$u$}}%

  \pgfxyline(0,5)(0,5.5)
  \pgfxyline(1.2,5)(1.2,5.5)

  \pgfxyline(0,4.25)(0,4.75)
  \pgfxyline(1.8,4.25)(1.8,4.75)        
  \ignorespaces}

\newcommand{\Slot}[1]{%
  \begin{pgftranslate}{\pgfpoint{#1}{0pt}}%
    \pgfsetlinewidth{0.6pt}%
    \color{structure}%
    \pgfmoveto{\pgfxy(-0.1,5.5)}%
    \pgfbezier{\pgfxy(-0.1,5.55)}{\pgfxy(-0.05,5.6)}{\pgfxy(0,5.6)}%
    \pgfbezier{\pgfxy(0.05,5.6)}{\pgfxy(0.1,5.55)}{\pgfxy(0.1,5.5)}%
    \pgflineto{\pgfxy(0.1,4.0)}%
    \pgfbezier{\pgfxy(0.1,3.95)}{\pgfxy(0.05,3.9)}{\pgfxy(0,3.9)}%
    \pgfbezier{\pgfxy(-0.05,3.9)}{\pgfxy(-0.1,3.95)}{\pgfxy(-0.1,4.0)}%
    \pgfclosepath%
    \pgfstroke%
  \end{pgftranslate}\ignorespaces}

\newcommand{\SlotZwei}[1]{%
  \begin{pgftranslate}{\pgfpoint{#1}{0pt}}%
    \pgfsetlinewidth{0.6pt}%
    \color{structure}%
    \pgfmoveto{\pgfxy(-0.1,5.5)}%
    \pgfbezier{\pgfxy(-0.1,5.55)}{\pgfxy(-0.05,5.6)}{\pgfxy(0,5.6)}%
    \pgfbezier{\pgfxy(0.05,5.6)}{\pgfxy(0.1,5.55)}{\pgfxy(0.1,5.5)}%
    \pgflineto{\pgfxy(0.1,4.25)}%
    \pgfbezier{\pgfxy(0.1,4.25)}{\pgfxy(0.05,4.15)}{\pgfxy(0,4.15)}%
    \pgfbezier{\pgfxy(-0.05,4.15)}{\pgfxy(-0.1,4.2)}{\pgfxy(-0.1,4.25)}%
    \pgfclosepath%
    \pgfstroke%
  \end{pgftranslate}\ignorespaces}

\newcommand{\ClipSlot}[1]{%
  \pgfrect[clip]{\pgfrelative{\pgfxy(-0.1,0)}{\pgfpoint{#1}{4cm}}}{\pgfxy(0.2,1.5)}\ignorespaces}

\newcommand{\ClipSlotZwei}[1]{%
  \pgfrect[clip]{\pgfrelative{\pgfxy(-0.1,0)}{\pgfpoint{#1}{4.25cm}}}{\pgfxy(0.2,1.25)}\ignorespaces}


\AtBeginSection[]{\frame{\frametitle{Outline}\tableofcontents[current]}}

\begin{document}

\frame{\titlepage}

%\section*{Outline}

\part{Parte I}
\frame{\frametitle{Outline}\tableofcontents[part=1]}
\end{document}
% \frame{ "All
%   things appear and disappear because of the concurrence of causes and
%   conditions. Nothing ever exists entirely alone; everything is in
%   relation to everything else.”

%   The Buddha (Gautama Siddharta, the founder of Buddhism, 563-483
%   B.C.)

% }
\section{Bases Filosóficas}


\frame{\frametitle{Mundo e Observador}
  \begin{tabular}{c c}
    \begin{minipage}{0.45\textwidth}
      \begin{description}
      \item[Realismo] Existe uma realidade independente do observador,
        que é percebida de forma direta pelos sentidos do observador.
      \item[Idealismo] Não existe uma realidade independendente do
        observador, a realidade é um produto da consciência do
        observador.
      \end{description}
    \end{minipage}&
    \begin{minipage}{0.45\textwidth}
      \includegraphics[scale=0.5]{magrite1}
    \end{minipage}
  \end{tabular}


}





\Frame{\frametitle{Determinismo}
  \begin{tabular}{c c}
    \begin{minipage}{0.45\textwidth}
      \begin{description}
      \item[Determinismo] A natureza obedece a leis imutáveis e
        universais, o futuro é consequência direta do passado. Eventos
        aleatórios são causados pela nossa ignorância.
      \item[Probabilismo] O presente não é consequência do passado, as
        leis da natureza apenas predizem a probabilidade de ocorrência
        de um determinado evento.
      \item[Intervencionismo] Existe uma ou várias entidades que
        interferem diretamente na natureza. Existem mecanismos que
        permitem aos homens interagirem com estas entidades.
      \end{description}
    \end{minipage}&
    \begin{minipage}{0.45\textwidth}
      \includegraphics[scale=0.5]{cause}
    \end{minipage}
  \end{tabular}


}

\frame{\frametitle{Localidade e não localidade}
  \begin{tabular}{c c}
    \begin{minipage}{0.45\textwidth}
      \begin{description}
      \item[Realidade Local] Existem entidades independentes, que
        interagem com outras entidades independentes de forma não
        instantânea.
      \item[Não localidade] Não existem entidades independentes, a
        natureza é formada por uma única entidade, eventos muito
        distantes podem interferir instantaneamente.
      \end{description}
    \end{minipage}&
    \begin{minipage}{0.45\textwidth}
      \includegraphics[scale=0.5]{naolocal}
    \end{minipage}
  \end{tabular}


}


\frame{\frametitle{Teorema de Bell} Ele garante condições bastante
  gerais que teoria físicas: realistas e locais, devem obedecer.  Os
  sistemas físicos em diversas situações violam estas condições, ou
  seja a natureza não pode ser descrita por uma teoria siultaneamente:
  \begin{itemize}
  \item local e
  \item realista.
  \end{itemize}

  A MQ não obedece estas condições e descreve os resultados
  experimentais.  Mas podem existir outras teorias mais gerais que ela
  e que possam explicar os mesmos resultados experimentais.

}

\frame{\frametitle{Resultados Recentes}
  \begin{itemize}
  \item Existem evidências experimentais que os estados quânticos
    possuam existência independente do observador.
  \item O que nos deixaria apenas com a possibilidade que a natureza
    seja não local.
  \end{itemize}
}

\frame{\frametitle{Visão de Einstein-Galileo}
  \begin{tabular}{c c}
    \begin{minipage}{0.45\textwidth}
      A natureza é descrita por leis relativamente simples que podem
      ser compreendidas de forma intuitiva, ou pelo menos que possam
      ser apreendidas pelo intelecto humano. A MQ é uma teoria
      provisória, que poderá ser abandonada quando uma teoria completa
      for descoberta.
    \end{minipage}&
    \begin{minipage}{0.45\textwidth}
      \includegraphics[scale=0.5]{einstein}
    \end{minipage}
  \end{tabular}
}


\frame{\frametitle{Pragmatismo} 
 \begin{tabular}{c c}
    \begin{minipage}{0.45\textwidth}
  Todas as teorias são provisórias,
  descrevem de forma imperfeita a natureza, eventualmenete estas
  teorias não são intuitivas, mas podemos manipulá-las de maneira
  formal e realizar predições. Nem todos os elemementos de uma teoria
  possuem contrapartida na natureza, mesmo nos casos em que esta
  descreve adequadamente os experimentos. 
    \end{minipage}&
    \begin{minipage}{0.45\textwidth}
      \includegraphics[scale=0.5]{magritecondhumana1933}
    \end{minipage}
  \end{tabular}

 }

\frame{\frametitle{Pragmatismo Dawkins} O nosso cérebro foi moldado
  pela evolução e está adaptado para viver no mundo em nossa escala de
  tempo e espaço, em escalas diferentes dessas esse aparato funciona
  de forma precária e deve ser substituído pelo formalismo matemático.
}

\frame{\frametitle{Mecânica Quântica - Conpenhagen}
  \begin{itemize}
  \item Formalimo matemático que permite realizar predições de
    resultados experimentais.
  \item Principal ingrediente $\Psi$: função de onda não pode ser
    medida ou percebida de forma direta.
  \item Não determinística e não local.
  \item Processo de medida interfere com objeto experimental.
  \end{itemize}
}

\frame{\frametitle{Observações}
 \begin{tabular}{c c}
    \begin{minipage}{0.45\textwidth}
  \begin{itemize}
  \item O fato de que o processo de medida interfere com o objeto não
    implica que o observador ``controle'' este objeto.
  \item objetos com energias altas o suficiente podem ser descritos
    com boa aproximação pela mecânica clássica
  \item existem vários efeitos quânticos relacionados a estrutura da
    matéria.
  \end{itemize}
    \end{minipage}&
    \begin{minipage}{0.45\textwidth}
      \includegraphics[scale=0.5, angle=90]{ParticleTracks}
    \end{minipage}
  \end{tabular}
}

\frame{\frametitle{MQ \& Sociedade}
  \begin{itemize}
  \item Transistores
  \item Criptografia Quântica
  \item Luz Laser
  \item $\ldots$
  \end{itemize}
}

\frame{\frametitle{MQ \& Sociedade}
  \begin{itemize}
  \item Homeopatia?
  \item Administração de Empresas?
  \item Cura Quântica?
  \item Telepatia?
  \item Universos Paralelos?
  \item Ovnis?
  \end{itemize}
}

\end{document}