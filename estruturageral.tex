% $Header: /cvsroot/latex-beamer/latex-beamer/examples/beamerexample5.tex,v 1.22 2004/10/08 14:02:33 tantau Exp $

\documentclass[11pt]{beamer}

\usetheme{Darmstadt}

\usepackage{times}
\usefonttheme{structurebold}

%\usepackage[english]{babel}
\usepackage[portuges]{babel}
\usepackage{pgf,pgfarrows,pgfnodes,pgfautomata,pgfheaps}
\usepackage{amsmath,amssymb}
%\usepackage[latin8]{inputenc}
\usepackage[utf8]{inputenc}
\usepackage{graphicx}

\setbeamercovered{dynamic}

\newcommand{\Lang}[1]{\operatorname{\text{\textsc{#1}}}}

\newcommand{\Class}[1]{\operatorname{\mathchoice
  {\text{\sf \small #1}}
  {\text{\sf \small #1}}
  {\text{\sf #1}}
  {\text{\sf #1}}}}

\newcommand{\NumSAT}      {\text{\small\#SAT}}
\newcommand{\NumA}        {\#_{\!A}}

\newcommand{\barA}        {\,\bar{\!A}}

\newcommand{\Nat}{\mathbb{N}}
\newcommand{\Set}[1]{\{#1\}}

\pgfdeclaremask{tu}{beamer-tu-logo-mask}
\pgfdeclaremask{computer}{beamer-computer-mask}
\pgfdeclareimage[interpolate=true,mask=computer,height=2cm]{computerimage}{beamer-computer}
\pgfdeclareimage[interpolate=true,mask=computer,height=2cm]{computerworkingimage}{beamer-computerred}
\pgfdeclareimage[mask=tu,height=.5cm]{logo}{logounesp}

\logo{\pgfuseimage{logo}}

\title{Estrutura Geral da Mecânica Ondulatória}
\author{Ney Lemke}
\institute[IBB-UNESP]{%
    Mec\^anica Qu\^antica}
\date{2013}                                

\colorlet{redshaded}{red!25!bg}
\colorlet{shaded}{black!25!bg}
\colorlet{shadedshaded}{black!10!bg}
\colorlet{blackshaded}{black!40!bg}

\colorlet{darkred}{red!80!black}
\colorlet{darkblue}{blue!80!black}
\colorlet{darkgreen}{green!80!black}

\def\radius{0.96cm}
\def\innerradius{0.85cm}

\def\softness{0.4}
\definecolor{softred}{rgb}{1,\softness,\softness}
\definecolor{softgreen}{rgb}{\softness,1,\softness}
\definecolor{softblue}{rgb}{\softness,\softness,1}

\definecolor{softrg}{rgb}{1,1,\softness}
\definecolor{softrb}{rgb}{1,\softness,1}
\definecolor{softgb}{rgb}{\softness,1,1}

\newcommand{\Bandshaded}[2]{
  \color{shadedshaded}
  \pgfmoveto{\pgfxy(-0.5,0)}
  \pgflineto{\pgfxy(-0.6,0.1)}
  \pgflineto{\pgfxy(-0.4,0.2)}
  \pgflineto{\pgfxy(-0.6,0.3)}
  \pgflineto{\pgfxy(-0.4,0.4)}
  \pgflineto{\pgfxy(-0.5,0.5)}
  \pgflineto{\pgfxy(4,0.5)}
  \pgflineto{\pgfxy(4.1,0.4)}
  \pgflineto{\pgfxy(3.9,0.3)}
  \pgflineto{\pgfxy(4.1,0.2)}
  \pgflineto{\pgfxy(3.9,0.1)}
  \pgflineto{\pgfxy(4,0)}
  \pgfclosepath
  \pgffill

  \color{black}  
  \pgfputat{\pgfxy(0,0.7)}{\pgfbox[left,base]{#1}}
  \pgfputat{\pgfxy(0,-0.1)}{\pgfbox[left,top]{#2}}
}

\newcommand{\Band}[2]{
  \color{shaded}
  \pgfmoveto{\pgfxy(-0.5,0)}
  \pgflineto{\pgfxy(-0.6,0.1)}
  \pgflineto{\pgfxy(-0.4,0.2)}
  \pgflineto{\pgfxy(-0.6,0.3)}
  \pgflineto{\pgfxy(-0.4,0.4)}
  \pgflineto{\pgfxy(-0.5,0.5)}
  \pgflineto{\pgfxy(4,0.5)}
  \pgflineto{\pgfxy(4.1,0.4)}
  \pgflineto{\pgfxy(3.9,0.3)}
  \pgflineto{\pgfxy(4.1,0.2)}
  \pgflineto{\pgfxy(3.9,0.1)}
  \pgflineto{\pgfxy(4,0)}
  \pgfclosepath
  \pgffill

  \color{black}  
  \pgfputat{\pgfxy(0,0.7)}{\pgfbox[left,base]{#1}}
  \pgfputat{\pgfxy(0,-0.1)}{\pgfbox[left,top]{#2}}
}

\newcommand{\BaenderNormal}
{%
  \pgfsetlinewidth{0.4pt}
  \color{black}
  \pgfputat{\pgfxy(0,5)}{\Band{input tapes}{}}
  \pgfputat{\pgfxy(0.35,4.6)}{\pgfbox[center,base]{$\vdots$}}
  \pgfputat{\pgfxy(0,4)}{\Band{}{}}

  \pgfxyline(0,5)(0,5.5)
  \pgfxyline(1.2,5)(1.2,5.5)
  \pgfputat{\pgfxy(0.25,5.25)}{\pgfbox[left,center]{$w_1$}}

  \pgfxyline(0,4)(0,4.5)
  \pgfxyline(1.8,4)(1.8,4.5)        
  \pgfputat{\pgfxy(0.25,4.25)}{\pgfbox[left,center]{$w_n$}}
  \ignorespaces}

\newcommand{\BaenderZweiNormal}
{%
  \pgfsetlinewidth{0.4pt}
  \color{black}
  \pgfputat{\pgfxy(0,5)}{\Band{Zwei Eingabeb\~AƒÂƒ\~A‚¤nder}{}}
  \pgfputat{\pgfxy(0,4.25)}{\Band{}{}}

  \pgfxyline(0,5)(0,5.5)
  \pgfxyline(1.2,5)(1.2,5.5)
  \pgfputat{\pgfxy(0.25,5.25)}{\pgfbox[left,center]{$u$}}

  \pgfxyline(0,4.25)(0,4.75)
  \pgfxyline(1.8,4.25)(1.8,4.75)        
  \pgfputat{\pgfxy(0.25,4.5)}{\pgfbox[left,center]{$v$}}
  \ignorespaces}

\newcommand{\BaenderHell}
{%
  \pgfsetlinewidth{0.4pt}
  \color{black}
  \pgfputat{\pgfxy(0,5)}{\Bandshaded{input tapes}{}}
  \color{shaded}
  \pgfputat{\pgfxy(0.35,4.6)}{\pgfbox[center,base]{$\vdots$}}
  \pgfputat{\pgfxy(0,4)}{\Bandshaded{}{}}

  \color{blackshaded}
  \pgfxyline(0,5)(0,5.5)
  \pgfxyline(1.2,5)(1.2,5.5)
  \pgfputat{\pgfxy(0.25,5.25)}{\pgfbox[left,center]{$w_1$}}

  \pgfxyline(0,4)(0,4.5)
  \pgfxyline(1.8,4)(1.8,4.5)        
  \pgfputat{\pgfxy(0.25,4.25)}{\pgfbox[left,center]{$w_n$}}
  \ignorespaces}

\newcommand{\BaenderZweiHell}
{%
  \pgfsetlinewidth{0.4pt}
  \color{black}
  \pgfputat{\pgfxy(0,5)}{\Bandshaded{Zwei Eingabeb\~AƒÂƒ\~A‚¤nder}{}}%
  \color{blackshaded}
  \pgfputat{\pgfxy(0,4.25)}{\Bandshaded{}{}}
  \pgfputat{\pgfxy(0.25,4.5)}{\pgfbox[left,center]{$v$}}
  \pgfputat{\pgfxy(0.25,5.25)}{\pgfbox[left,center]{$u$}}%

  \pgfxyline(0,5)(0,5.5)
  \pgfxyline(1.2,5)(1.2,5.5)

  \pgfxyline(0,4.25)(0,4.75)
  \pgfxyline(1.8,4.25)(1.8,4.75)        
  \ignorespaces}

\newcommand{\Slot}[1]{%
  \begin{pgftranslate}{\pgfpoint{#1}{0pt}}%
    \pgfsetlinewidth{0.6pt}%
    \color{structure}%
    \pgfmoveto{\pgfxy(-0.1,5.5)}%
    \pgfbezier{\pgfxy(-0.1,5.55)}{\pgfxy(-0.05,5.6)}{\pgfxy(0,5.6)}%
    \pgfbezier{\pgfxy(0.05,5.6)}{\pgfxy(0.1,5.55)}{\pgfxy(0.1,5.5)}%
    \pgflineto{\pgfxy(0.1,4.0)}%
    \pgfbezier{\pgfxy(0.1,3.95)}{\pgfxy(0.05,3.9)}{\pgfxy(0,3.9)}%
    \pgfbezier{\pgfxy(-0.05,3.9)}{\pgfxy(-0.1,3.95)}{\pgfxy(-0.1,4.0)}%
    \pgfclosepath%
    \pgfstroke%
  \end{pgftranslate}\ignorespaces}

\newcommand{\SlotZwei}[1]{%
  \begin{pgftranslate}{\pgfpoint{#1}{0pt}}%
    \pgfsetlinewidth{0.6pt}%
    \color{structure}%
    \pgfmoveto{\pgfxy(-0.1,5.5)}%
    \pgfbezier{\pgfxy(-0.1,5.55)}{\pgfxy(-0.05,5.6)}{\pgfxy(0,5.6)}%
    \pgfbezier{\pgfxy(0.05,5.6)}{\pgfxy(0.1,5.55)}{\pgfxy(0.1,5.5)}%
    \pgflineto{\pgfxy(0.1,4.25)}%
    \pgfbezier{\pgfxy(0.1,4.25)}{\pgfxy(0.05,4.15)}{\pgfxy(0,4.15)}%
    \pgfbezier{\pgfxy(-0.05,4.15)}{\pgfxy(-0.1,4.2)}{\pgfxy(-0.1,4.25)}%
    \pgfclosepath%
    \pgfstroke%
  \end{pgftranslate}\ignorespaces}

\newcommand{\ClipSlot}[1]{%
  \pgfrect[clip]{\pgfrelative{\pgfxy(-0.1,0)}{\pgfpoint{#1}{4cm}}}{\pgfxy(0.2,1.5)}\ignorespaces}

\newcommand{\ClipSlotZwei}[1]{%
  \pgfrect[clip]{\pgfrelative{\pgfxy(-0.1,0)}{\pgfpoint{#1}{4.25cm}}}{\pgfxy(0.2,1.25)}\ignorespaces}


\AtBeginSection[]{\frame{\frametitle{Outline}\tableofcontents[current]}}

\begin{document}

\frame{\titlepage} \frame{\frametitle{Resumo}

$$i\hbar\frac{\partial \psi}{\partial t}=\hat{H}\psi \quad 
\hat{H}=\frac{\hat{p^2}}{2m}+V(x)$$

onde:

$$\hat{p}=-i\hbar\frac{\partial}{\partial x}$$

$$\psi(x,t)=u_E e^{-iEt/\hbar} \quad \hat{H}u_E(x)=Eu_E(x)$$
}

\frame{\frametitle{Observações}
\begin{itemize}
    \item $$\int_{-\infty}^\infty dx u_{E_1}^*(x)u_{E_2}(x) =\delta_{E_1,E_2}$$
    o valor $a$.
  \item Uma vez que o sistema é medido o pacote de onda colapsa e em
    medidas imediatamente subsequentes o observável $A$ valerá $a$.
  \end{itemize}
}

\frame{\frametitle{Teorema:}
$$\sum_a |C_a|^2=1$$
$$\int_{-\infty}^\infty dx \psi^*\psi=1$$
$$\int dx \psi^*\sum_a C_a u_a$$
$$ \sum_{a,a^\prime} C_aC^*_{a^\prime} \int dx u_au_{a^\prime}^*=
\sum_{a,a^\prime} C_aC^*_{a^\prime}\delta_{aa^\prime}=\sum_a
|C_a|^2=1$$ }

\frame{\frametitle{Teorema:}
$$\sum_a u_a^*(x)u_a(y)=\delta (x-y)$$

$$\sum_a C_a^*C_a=\sum_a\int_{-\infty}^\infty dx\psi(x)u^*_a(x)  
\int_{-\infty}^\infty dy \psi^*(y)u_a(y)=1$$
$$=\int_{-\infty}^\infty dx\int_{-\infty}^\infty dy \psi^*(y)\psi(x) \sum_a u_a(x) u_a(y)=1$$
A única forma de satisfazermos essa igualdade é se:
$$\sum_a u_a^*(x)u_a(y)=\delta (x-y)$$
}

\frame{\frametitle{Operador Conjugado}

$$\int_{-\infty}^\infty dx \psi^* Q^\dagger \psi =\int_{-\infty}^\infty dx (Q\psi(x))^*\psi $$

Operador Auto-Adjunto:

$$Q^\dagger=Q$$


}

\frame{\frametitle{Teorema}
$$(AB)^\dagger=B^\dagger A^\dagger$$

$$\int dx (AB\phi )^* \psi=\int dx (A \chi)^*\psi=\int dx \chi^* A^\dagger \psi$$

$$=\int dx \phi^*
 B^\dagger A^\dagger \psi$$
}


\frame{\frametitle{Teorema}


$$(\Delta A)^2(\Delta B)^2\geq \frac{1}{4}\left\langle i[A,B] \right\rangle ^2$$

Este resultado mostra que que se $A$ e $B$ não comutam eles não podem
ser medidos simultaneamente.

}


\frame{\frametitle{Degenerescência e Observáveis Simultâneos} {\bf
    Teorema:} Se $u_a$ são autofunções de $A$ e $B$ simultaneamente
  para qualquer $a$ então $[A,B]=0$.

$$Au_a=au_a\quad Bu_a=bu_a$$

$$ABu_a=abu_a=bau_a=BAu_a$$

Logo:

$$AB=BA$$
}

\frame{\frametitle{Degenerescência e Observáveis Simultâneos} Se
  $[A,B]=0$ as autofunções de $A$ são também autofunções de $B$ se as
  autofunções são não degeneradas.

$$ABu_A=BAu_A=aBu_A$$

Se as autofunções de $A$ são não degeneradas temos que:

$$Bu_a=bu_a$$
O que acontece no caso degenerado?  }

\frame{\frametitle{Caso degenerado}


$$Au_A=au_a^1\quad Au_a^2=au_A^2$$

$$Bu_a^1=b_{11}u_a^1+b_{12}u_a^2 \quad
Bu_a^2=b_{21}u_a^1+b_{22}u_a^2$$

Escolhemos então $v_a^1$ e $v_a^2$ tais que:

$$Av_{ab}=av_{ab} \quad Bv_{ab}=bv_{ab}$$
}

\frame{\frametitle{Caso degenerado} Se houver ainda um terceiro
  operador que comute com $A$ e $B$, $C$ neste caso podemos extender
  esse processo e definir:

$$Aw_{abc}=aw_{abc} \quad Bw_{abc}=bw_{abc}$$

$$Cw_{abc}=cw_{abc}$$

}

\frame{\frametitle{Caso degenerado} Neste caso dizemos que os
  observáveis $A$, $B$ e $C$ formam um conjunto completo de
  observáveis que comutam (admitimos que as autofunções $w_{abc}$ são
  não degeneradas.

  Os observáveis $A$ e $B$ e $C$ são as quantidades que podemos
  conhecer simultâneamente de um determinado sistema físico.

  Exemplo: $E$, $L^2$ e $L_z$ no caso do átomo de hidrogênio.

}


\frame{\frametitle{Dependência Temporal}

$$\langle A_t\rangle = \int \psi^*(x,t) A(t) \psi(x,t)$$

$$\frac{d}{dt}\langle A\rangle =\int dx \left[ \frac{\partial \psi^*}{\partial t}A\psi+
  \psi^*\frac{\partial A}{\partial t} \psi + \psi^* A \frac{\partial \psi}{\partial
    t}\right]$$

$$=\left\langle \frac{\partial A}{\partial t}\right\rangle +
\int dx \left[ \left(\frac{1}{i\hbar} H\psi \right)^*A\psi\right]
+\left[ \psi^* A \left( \frac{1}{i\hbar} H\psi \right) \right]
$$

$$=\left\langle \frac{\partial A}{\partial t}\right\rangle+
\frac{1}{i\hbar}\int dx\left[ -\psi^*HA\psi+\psi^*AH\psi\right]$$
 $$=\left\langle \frac{\partial A}{\partial t}\right\rangle+
 \frac{i}{\hbar}\left\langle [H,A] \right\rangle$$ }

\frame{\frametitle{Dependência Temporal $\langle x\rangle$}
$$\hat{H}=\frac{\hat{p}^2}{2m}+V(x)$$

$$\frac{d \langle x\rangle}{dt} =\frac{i}{\hbar}\langle [H,x] \rangle $$

$$=\frac{i}{\hbar}\left\langle \left[\frac{p^2}{2m}+V(x),x\right]\right\rangle$$

$$=\frac{i}{\hbar}\left\langle \left[ \frac{p^2}{2 m},x\right]\right\rangle$$
}

\frame{\frametitle{Dependência Temporal $\langle x\rangle$}
$$\left[ \frac{p^2}{2m},x \right]=\frac{1}{2m}[p^2,x]$$

$$\frac{1}{2m} (p^2x -xp^2+pxp-pxp)$$

$$\frac{1}{2m}(p[p,x]+[p,x]p)=\frac{\hbar}{im}p$$

$$\frac{d \langle x\rangle}{dt} = \frac{\langle p \rangle}{m} $$

}

\frame{\frametitle{Dependência Temporal $\langle p\rangle$}

$$\frac{d \langle p\rangle}{dt}=\frac{i}{\hbar}\left\langle\left[ \frac{p^2}{2m}+V(x),p\right]\right\rangle =\frac{i}{\hbar}\left\langle [V(x),p] \right\rangle$$

$$\frac{\hbar}{i}V\frac{\partial\psi}{\partial x}-\frac{\hbar}{i}\frac{\partial}{\partial x}(V\psi)$$

$$\frac{\hbar}{i}V\frac{\partial \psi}{\partial x}-\frac{\hbar}{i}\frac{\partial V}{\partial x}-\frac{\hbar}{i}V\frac{\partial \psi}{\partial x}
=i\hbar\frac{\partial V}{\partial x}$$

$$\frac{d \langle p\rangle}{d t}=-\left\langle \frac{\partial V}{\partial x} \right\rangle $$


}

\frame{\frametitle{Limite Clássico}

$$m\frac{d^2 \langle x\rangle}{dt^2}=-\left\langle \frac{\partial V}{\partial x}\right\rangle $$

Note que:

$$\left\langle \frac{dV}{dx}\right\rangle\neq \frac{d\langle V (\langle x\rangle )}{d\langle x \rangle }$$

$$F(x)=F(\langle x\rangle )+(x-\langle x \rangle )F^\prime (\langle x\rangle )$$

$$F(x)\sim F(\langle x\rangle )$$

}
\end{document}
